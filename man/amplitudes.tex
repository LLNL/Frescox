%***********************************************************************
% 
%    Copyright 2018, I.J. Thompson
%
%    This file is part of FRESCOX.
%
%    FRESCOX is free software: you can redistribute it and/or modify it
%    under the terms of the GNU General Public License as published by
%    the Free Software Foundation, either version 3 of the License, or
%    (at your option) any later version.
%
%    FRESCOX is distributed in the hope that it will be useful, but
%    WITHOUT ANY WARRANTY; without even the implied warranty of
%    MERCHANTABILITY or FITNESS FOR A PARTICULAR PURPOSE. See the
%    GNU General Public License for more details.
%
%    You should have received a copy of the GNU General Public License
%    along with FRESCOX. If not, see <http://www.gnu.org/licenses/>.
%
%    OUR NOTICE AND TERMS AND CONDITIONS OF THE GNU GENERAL PUBLIC
%    LICENSE
%
%    The precise terms and conditions for copying, distribution and
%    modification are contained in the file COPYING.
%
%***********************************************************************
\documentclass[11pt,a4paper]{article}
\input wider.sty
\parindent 0pt
\parskip 10pt
\newcommand{\vecr}{{\bf r}}
\newcommand{\veck}{{\bf k}}
\newcommand{\vecR}{{\bf R}}
\newcommand{\vecl}{{\bf l}}
\newcommand{\vecL}{{\bf L}}
\newcommand{\vecJ}{{\bf J}}
\newcommand{\vecs}{{\bf s}}
\newcommand{\vecx}{{\bf x}}
\newcommand{\Vee}{{\sf V}}
\newcommand{\brho}{{\mbox{\boldmath $\rho$}}}
\newcommand{\half}{\frac{1}{2}}
\begin{document}


\subsection*{Scattering Amplitudes}

%%.rc 5 on
The Rutherford amplitude for pure Coulomb scattering
(with no $ e^{2i \sigma_0} $ factor) is
\begin{eqnarray} \label{ruther}
F_c ( \theta ) = - {\eta \over 2k} ~~
               { \exp (-2 i \eta \ln(\sin \theta /2))  \over  \sin^2 \theta /2}
\end{eqnarray}
%%.rc 5 off
The Legendre coefficients for the scattering to the projectile state
$ J'_ p $ and target state $ J'_ t $
from initial projectile state $ J_p $
and target state $ J_t $ are given by
\begin{eqnarray}  \nonumber
 A^{L'}_{m' M' ;mM} &=&
   \sum_{L,J,J' ,J_T }
    \langle L0 J_p m | Jm\rangle\langle Jm J_t M |J_T M_T\rangle \\
&& \langle L' M_{L'} J'_ p m' | J' M_{L'} + m'\rangle
    \langle J' M_{L'}  +m' J'_ t M' |J_T M_T\rangle \nonumber\\
&&  \nonumber
  {4 \pi \over k} \sqrt {\frac{k'}{\mu'}\frac{\mu}{k}}
    e^{i( \sigma_L - \sigma_0 )}
      e^{i( \sigma'_ {L'} - \sigma'_ 0 )}\\
&&  \label{ALeg}
  \left ( {i \over 2} \right )
      \left [ \delta_{\alpha ,\alpha'} - S^{J_T}_{\alpha ,\alpha'} \right ]
    \sqrt {{2L+1 \over {4 \pi}}}~ Y_c (L' ,M_{L'})
\end{eqnarray}
where $Y_c (L,M)$ is the coefficient of
$P_L^{|M|} (\cos \theta )
    e^{iM\phi}$ in $Y_L^M (\theta ,\phi )$,
$  \sigma_L = \arg \Gamma (1 + L + i \eta )$ is the Coulomb phase shift,
$\alpha'$ refers to the primed values
$ L'  J'_ p J'_ t  k' \mu' $ etc.,
and $\alpha$ refers to the unprimed values
$ L J_p J_t k \mu $.

For each outgoing channel $J'_ p , J'_ t $,
we may then calculate the angular-dependent scattering amplitudes
\begin{eqnarray}
f_{m' M' : mM} (\theta) =
    \delta_{J_p , J'_ p}
    \delta_{J_t , J'_ t}
     F_c (\theta) +
  \sum_{L'} A_{m' M' : mM}^{L'}
                    P_{L'}^{m' +M' -m-M} (\cos \theta)
\end{eqnarray}
in terms of which the differential cross section is
\begin{eqnarray}
  {d \sigma(\theta) \over d \Omega} =
   {1 \over (2J_p + 1)(2J_t + 1) }
   \sum_{m' M' m M}
           \left | f_{m' M' : mM} (\theta) \right | ^2 .
\end{eqnarray}


\subsection*{Improvements?}
Can we avoid the excessive recalculation of the second two Clebsch-Gordan coefficients in eq.(\ref{ALeg})?
Do they really have to be recalculated for every $m' M' ;mM'$ and $\alpha,\alpha'$?

\end{document}

