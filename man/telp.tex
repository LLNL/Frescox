%***********************************************************************
% 
%    Copyright 2018, I.J. Thompson
%
%    This file is part of FRESCOX.
%
%    FRESCOX is free software: you can redistribute it and/or modify it
%    under the terms of the GNU General Public License as published by
%    the Free Software Foundation, either version 3 of the License, or
%    (at your option) any later version.
%
%    FRESCOX is distributed in the hope that it will be useful, but
%    WITHOUT ANY WARRANTY; without even the implied warranty of
%    MERCHANTABILITY or FITNESS FOR A PARTICULAR PURPOSE. See the
%    GNU General Public License for more details.
%
%    You should have received a copy of the GNU General Public License
%    along with FRESCOX. If not, see <http://www.gnu.org/licenses/>.
%
%    OUR NOTICE AND TERMS AND CONDITIONS OF THE GNU GENERAL PUBLIC
%    LICENSE
%
%    The precise terms and conditions for copying, distribution and
%    modification are contained in the file COPYING.
%
%***********************************************************************
\documentclass{article}

\begin{document}

\subsection*{Average local polarisation potential}

We can define a complex ``weighted mean'' local polarisation
potential $V^{P}(R)$  [1]

\begin{equation}
V^{P}(R)=\frac{\sum_{J}w_{J}(R)V_{J}^{TE}(R)}{\sum_{J}w_{J}(R)},  \label{Eq13}
\end{equation}
where $V_{J}^{TE}(R)$ are the ``trivially equivalent potentials'' defined by
\begin{equation}
V_{J}^{TE}(R)=\frac{1}{f_{g.s, J}(R)}\sum_{\alpha^{\prime}\neq g.s} V_{g.s:\alpha^{\prime}}^{J}(R)f_{\alpha^{\prime}, J}(R),  \label{Eq14}
\end{equation}
and $w_{J}(R)$ are weight factors chosen as
\begin{equation}
w_{J}(R)=a_{J}\mid f_{g.s, J}(R)\mid ^{2},  \label{Eq15}
\end{equation}
for some coefficients $a_{J}$ to be specified. This choice of weight factors
$w_{J}(R)$ avoids singularities when $f_{g.s, J}(R)$ has a node in
some $R$. The coefficients $a_{J}$ are considered as proportional to partial
reaction cross sections
\begin{equation}
a_{J}=(2J+1)(1-\mid S_{J}\mid ^{2}),  \label{Eq16}
\end{equation}
where $S_{J}$ are the elastic S-matrix elements for each $J$ value.
A single channel calculation (elastic
channel) using the sum of ``$V_{g.s:g.s}^{J}(R)$ + $V^{P}(R)$'' should
approximately reproduce the elastic scattering [1]) cross sections.

\bigskip
\noindent
[1] I.J. Thompson, M.A. Nagarajan, J.S. Lilley and M.J. Smithson,
Nucl. Phys. A 505 (1989) 84.

\end{document}
