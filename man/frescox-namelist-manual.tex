%***********************************************************************
% 
%    Copyright 2020, I.J. Thompson
%
%    This file is part of FRESCOX.
%
%    FRESCOX is free software: you can redistribute it and/or modify it
%    under the terms of the GNU General Public License as published by
%    the Free Software Foundation, either version 2 of the License, or
%    (at your option) any later version.
%
%    FRESCOX is distributed in the hope that it will be useful, but
%    WITHOUT ANY WARRANTY; without even the implied warranty of
%    MERCHANTABILITY or FITNESS FOR A PARTICULAR PURPOSE. See the
%    GNU General Public License for more details.
%
%    You should have received a copy of the GNU General Public License
%    along with FRESCOX. If not, see <http://www.gnu.org/licenses/>.
%
%    OUR NOTICE AND TERMS AND CONDITIONS OF THE GNU GENERAL PUBLIC
%    LICENSE
%
%    The precise terms and conditions for copying, distribution and
%    modification are contained in the file COPYING.
%
%***********************************************************************
\documentclass[11pt]{article}
\usepackage{times}
\usepackage{mathptm}
\topmargin 0cm
\oddsidemargin -1.0cm \evensidemargin -1.0cm
\textheight 22.0cm \textwidth 17.00cm
%
\title{\Huge Frescox}
\date{Version FRESCOX-7.1, June 2020\\ LLNL-CODE-811517 }
\author{I.J. Thompson
%
\\Nuclear Theory and Modeling Group
\\PO Box 808, L-414,
 Livermore, CA 94551, USA
\\Telephone + 1 (925) 423 4978
\\ %
\\Email: I-Thompson@llnl.gov
\\Available: https://github.com/LLNL/Frescox
\\More information: http://www.fresco.org.uk }


\parskip 0pt
\parindent 0pt

\newcommand{\beqn}{\begin{eqnarray}}
\newcommand{\eeqn}{\end{eqnarray}}
\newcommand{\upq}{\textsf{'}}
%
\begin{document}
\maketitle
%\centerline{\large Fresco}
\tableofcontents
\newpage
\section{Introduction}

\parskip 5pt
\parindent 10pt

FRESCOX is a Coupled-Channels program that can have finite-range
transfer interactions among any number of mass partitions, and any
number of nuclear excitations in each partition.

This is accomplished by calculating and storing the kernels of the
non-local interactions, and then solving the coupled-channels set
iteratively.
At each iteration the stored kernels are used to integrate the wave
functions to generate the source terms for the next iteration.
Pade approximants can be used if the iterations diverge because of
strong couplings.

The program can also generate local couplings for the rotational
or single-particle excitations of either the projectile or the
target in any partition,
the local form factors for zero-range or local-energy-approximation
can also be used.
These various local interactions may also be included iteratively
(to give multistep DWBA), or alternatively the first few coupled
channels may be blocked together and solved by exact coupled-channels
methods.

Special treatment is provided for any long ranged Coulomb multipoles,
using James Christley's coupled-Coulomb functions CRCWFN,
and interpolation in partial waves is also possible.

The nonlocal kernels for single-particle transfers are calculated first
at a much smaller number of interpolation points,
and then expanded when necessary to calculate the source terms by
integrating
\beqn
S(R _ f ) = \int _ 0 ^ {R _ {match}}
                K _ {fi} (R _ f , R _ i ) u(R _ i ) dR _ i
\eeqn
where RMATCH and HCM, the step size, are given in the \&Fresco namelist.
Since the kernel function $K_{fi}(R_f,R_i)$ is usually rapidly varying with
$D_{fi} = R_{f} - R_{i}$ (especially with heavy-ion reactions), and only slowly
varying with $R_{f}$ (if $D_{fi}$ is constant),  FRESCOX calculates and stores
the function $K_{fi}'(R_{f},D_{fi})$ at intervals of RINTP in $R_{f}$,
and intervals of HNL in $D_{fi}$. The $D_{fi}$ range considered is
CENTRE-RNL/2 to CENTRE+RNL/2, i.e. range of RNL centred at CENTRE,
and FRESCOX later suggests improved values for RNL \& CENTRE.
The HNL reflects to physical variation of $K_{fi}'$ with $D_{fi}$, and can be
a fraction of HCM (for heavy ion reactions) or a multiple of HCM
(for light ion reactions especially with `prior' interactions).
If HNL is a submultiple of HCM, the program anticipates the interpolation
of $u(R_{i})$ in the equation above, and only stores an effective kernel
function at intervals of HCM.

The new variable MTMIN in the FRT version controls the method used to
calculate the form factors of the transfer kernels. If the L-transfer value
(estimated by $\ell_f + \ell_i$)
is larger than or equal to MTMIN, then the m-dependent expressions for the
spherical harmonics are used explicitly.
Otherwise, the
angular dependencies of the initial and final transfer states
are transformed into those of the channel variables $R_{i}$ \& $R_{f}$ using
Moshinsky's solid-harmonic transformation, as given for example
in
%\footnote{
Austern et al. Phys. Rev. {\bf 133} pp B 3 - 16.
%}
%
By careful control of the numerical approximations, the second method can
be extended to find heavy-ion transfer form factors, but the maximum
transferred angular momentum L is still limited by numerical cancellation
errors arising from the finite roundoff errors in the computer.
With 8-byte word lengths, the unit roundoff is approximately $10^{-15}$,
and this limits the transferred L to 6 or less, so the default value
of MTMIN is set to 6. The Moshinsky method is much faster when it is
accurate, so the default value of MTMIN has been set as high as practical.

A general description of the formalisms used is given in
%\footnote{
I.J. Thompson,
{\em Coupled Reaction Channels Calculations in Nuclear Physics},
Computer Physics Reports, {\bf 7} (1988) pp 167 - 212.
%}

%}
% .
\newpage
\section{Input card-image format}
Please see document   frescox-input-manual.pdf.
\section{Input NAMELIST format}
The namelist input has two fixed-format lines to start with, and these followed by the namelists in Fortran style free-format. 
Variable names may be upper or lower case without distinction, but in this manual they are all capitalized.
All namelists are terminated by a {\bf /}. Some systems trailing comments begun by a {\bf !}.
%In this manual, Fresco variable names are written in {\sc bold}, and literal strings are in {\tt monospaced font}.

\subsection{Initial Lines}
\begin{description}
\item[Header line] \hfill 

One line of text to describe the nature of this run, for user's information only. (120 characters max)
\item[NAMELIST] \hfill 

Fixed string to indicate input format using namelists
\end{description}

\subsection{The \&FRESCO namelist}

\subsubsection*{Complete list of variables}

{\em Radial coordinates:}
hcm, rmatch, rintp, hnl, rnl, centre, hnn, rnn, rmin, rsp, cutl, cutr, cutc, \\
rasym, accrcy, switch, sinjmax, ajswtch, 

\noindent
{\em Partial waves:}
jtmin, jtmax, absend, jleast,  jump, jbord, pset, jset,  iso, llmax, 



\noindent
{\em Angular distributions:}
kqmax,  thmin, thmax, thinc, pp, koords,  nearfa, dist0,

\noindent
{\em Defining coupled equations:}
inh, nnu, maxl, minl, mtmin, epc,  erange, dk,  plane,  elpmax, rela, relref,  \\
ccbins, complexbins, sumform, ompform,   pluto,  unitmass, finec,  

%  obsolete variables: sock, gscouplonly,  masfil, 
%  not to be changed; hktarg 

\noindent
{\em Incident channel:}
 pel, exl, lab, lin, lex, elab, nlab, 

\noindent
{\em Solving equations:}
ips, it0, iter, fatal, iblock, pade, nosol, dry, smallchan, smallcoup, ccreal,  \\
psiren, initwf, mpihelp,  maxcoup, expand, 


\noindent
{\em R-matrix setup}:
nrbases, nlagcc, nrbmin, pralpha, pcon, rmatr, btype, bndx, buttle, weak, \\
krm, eigens, meigs, vsearch, echan, enodes, nparameters, eobs  

\noindent
{\em Trace variables:}
chans, listcc, treneg, cdetr, smats,    

\noindent
{\em Output details:}
xstabl, nlpl, waves, lampl, veff, kfus, nfus, wdisk, bpm, melfil, cdcc,  tcfile, tcfilename,  \\
dist0, tmp, grace, boldplot


\subsubsection{Radial coordinates}

\begin{description}
\item[HCM,  RMATCH, RINTP, HNL, RNL, CENTRE, HNN, RNN, RMIN, RSP, CUTL, CUTR, CUTC] \hfill

%rnn,rmin,rsp, sinjmax,plane,
%pset,jset,jleast,
%kqmax,pp,thmin,thmax,thinc,koords,cutl,cutr,cutc,complexbins,
%ips,it0,iter,fatal,iblock,pade,psiren,iso,nnu,maxl,minl,mtmin,
%epc,erange,dk, nosol,nrbases,nrbmin,pralpha,pcon,rmatr,bndx,
%meigs,buttle,weak,llmax,expand,hktarg,mpihelp,eigens,nlagcc,
%chans,listcc,treneg,cdetr,smats,xstabl,nlpl,waves,ccreal,
%lampl,veff,kfus,wdisk,bpm,melfil,cdcc,nfus, nparameters,
%smallchan,smallcoup,sumform,maxcoup,ompform,initwf,pluto,
%inh,TMP,MASFIL,unitmass,finec,pel,exl,lab,lin,lex,elab,nlab,
%vsearch,echan,enodes,gscouplonly,ccbins,sock,btype,boldplot,
%KRM, dist0,tcfile,dgam,eobs,grace,elpmax,relref

Wave functions calculated at intervals of HCM up to abs(RMATCH).

If RMATCH $<$ 0, then use values of RASYM, ACCRCY, SWITCH, AJSWITCH for  coupled Coulomb wave functions.

Non-local kernels $K_{fi}'(R_{f},D_{fi})$ calculated at $R_f$ intervals of RINTP,
and for a non-local ($D_{fi}$) range of RNL centred at CENTRE in steps of HNL.

RMATCH and RINTP are rounded to multiples of HCM, and HNL is rounded either to
a multiple or a sub-multiple of HCM.

For two-nucleon transfers, the nucleon-nucleon distance is
discretized from RMIN to RNN in a multiple of 6 Gaussian quadrature
points to give step size close to HNN.

RSP is the upper limit of state radius when folding single-particle states (bound states or continuum bins) with
KIND=3 or 4 couplings.


CUTL: set the number of radial points per $\ell$ of the lower radial cutoff when integrating the radial equations. Default = --1.6\\
When CUTL$>$0, use $\ell=J$ (total angular momentum of CC set),\\
When CUTL$<$0, use $\ell=L_{in}$ (orbital angular momentum of incoming partial wave). 
Using  CUTL$<$0 gives more accurate analyzing powers.

CUTR = lower radial cutoff (fm).  Combining with CUTL is to use $R_{\min}$ = max(CUTL*$\ell$*HCM,CUTR).\\
If CUTR $<$ 0, put cutoff at point-Coulomb turning point $- |$CUTR$|$.

CUTC = lower radial cutoff (in fm) for off-diagonal couplings.

\end{description}

%
\begin{description}
\item[RASYM, ACCRCY, SWITCH, SINJMAX, AJSWITCH] \hfill


Use coupled Coulomb wave functions from CRCWFN out to asymptotic radius RASYM from
inner radius abs(RMATCH) for those partitions in which PWF is TRUE.
% Otherwise, use uncoupled Coulomb wavefunctions at radius abs(RMATCH).
\\
If RASYM $<$ 0, then determine the outer radius in order that classical
Coulomb trajectories reach an angle abs(RASYM) degrees.


ACCRCY is an accuracy parameter controlling the piecewise step length.
Default is 0.01: smaller values give greater accuracy.


SWITCH is the radius at which to switch from Airy functions
to sines and cosines in piecewise method.  Default is 1000 fm.
If SINJMAX$>0$, then change switchover condition to $J>$ SINJMAX.


AJSWITCH
Normally the Coupled Coulomb wfns are matched to zero
and the Numerov integration is omitted
when the Coulomb distance of closest approach is more than 4.5 fm outside
abs(RMATCH) (or  the  --CUTR distance if   CUTR negative).
This is only allowed when $J \geq $AJSWITCH.  Default is 0.0
\end{description}

\subsubsection{Partial waves}


\begin{description}
%\item[JTMIN,JTMAX,ABSEND,DRY,CSET, RELA, NEARFA, JUMP(:),JBORD(:)]

\item[JTMIN, JTMAX, ABSEND, JLEAST]
Calculate coupled-channels sets with total $J$ in the interval
max(0,JTMIN) $ < J  <$ JTMAX,
stopping sooner if the absorption from the elastic channel
is less than ABSEND millibarns of three successive $J$/parity sets.
(If ABSEND $<$ 0, this takes the full $J$ interval.)\\
If JLEAST$>0$, then only stop if $J>$ JLEAST.
\\
If JTMIN $<$ 0, then for $J <$ abs(JTMIN) include only the incoming channel
in the calculations, ignoring transfers and excited states.
This is needed if the elastic scattering cross sections are to be given
correctly.



\item[JUMP(1:), JBORD(1:)]
Calculate coupled-channels sets not for every $J$
value, but at intervals of JUMP(i) for  $J\geq$ JBORD(i), for i=1,5.
(The program sets JUMP(0)=1 \& JBORD(0)=JTMIN, and, 
if there are NJ non-zero values, also JBORD(NJ+1)=JTMAX.  
This gives no $J$ jumping if JUMP and JBORD are not set in the namelist.)
The omitted $J$ values are provided by interpolation on the scattering
amplitudes $A(m'M':mM; L)$ prior to calculating cross sections.

\item[PSET, JSET] ~\\
If PSET =  --1 or +1,  restrict parity of CRC sets to that value (0 = no restriction)\\
If PSET = --2 or +2, restrict CRC sets to unnatural or natural parities respectively. Here, `natural parity' is defined as parity = (--1)**nint($J+.1$). 

JSET = number of CRC sets to calculate before stopping (0 = all sets)

\item[ISO]
\ = `A' or `J', for replacing all barriers $L(L+1)$ by that for $L = J$ barrier
\\ = `B' or `L', for replacing all barriers by $L = L_{in}$ barrier.
 \\ = 0 or blank, for no isocentrifugal approximations,

Of course, the {\em simplest and fastest} way to use the isocentrifugal
approximation is to put all spins and parities to $0^+$, and all
transition multipoles to $k=0$. The ISO variable is not needed then.

\item[LLMAX] Maximum partial wave $L$ in any CRC set.
\end{description}


\subsubsection{Angular distributions}

\begin{description}

\item[KQMAX, THMIN, THMAX, THINC]

Give cross sections (and tensor analyzing powers up to rank K = KQMAX)
for centre-of-mass scattering angle from THMIN
to abs(THMAX) in steps of THINC.\\
Elastic channels normally output the ratio to Rutherford, unless THMAX $< 0$.

\item[PP]
Calculate analyzing powers/polarizations for projectile (PP=0 or blank),
target (PP=1), ejectile (PP=2) or residual nucleus (PP=3).
PP=4 gives projectile (PP=0) analyzing powers, along with Kyy results.

\item[KOORDS ]determines the coordinate systems used for the analyzing powers:
\\  = 0 : Madison coordinates (default)
\\  = 1 : Madison + Transverse
\\  = 2 : Madison + Transverse + Recoil
\\  = 3 : Madison + Transverse + Recoil + Hooton-Johnson

\item[NEARFA]
For mod(NEARFA,10) values: 
\\ = 0 or 1 for the usual cross sections,
\\  = 2 or --2 for printing `far side' cross sections too,
\\  = 3 or --3 for printing far and near side cross sections too.
\\  $>$ 0 for printing far \& near-sides for elastic channel only.
\\  $<$ 0 for printing far \& near-sides for all channels.\\
If abs(NEARFA)$>$10, also split the Coulomb amplitude according to Cha, CPC 176 (2007) 318.

\end{description}

\subsubsection{Defining coupled equations}

\begin{description}

\item[INH]
 = 0 : zero-range transfer forms in intervals of HCM exactly
\\ = 1 : stored in steps of HCM * (proj. core)/(proj. composite mass)
\\ = 2 : stored in steps of HCM * (targ. core)/(targ. composite mass)\\
    So INH=2 corrects for longitudinal recoil during transfers with zero-range projectiles.

\item[NNU, MAXL, MINL, MTMIN, EPC] control the accuracy of non-local transfer form factors:

NNU is the number of Gaussian integration points
in the angular
integration used for the non-local transfer kernels. NNU should
be a multiple of 6; NNU = 18 is the minimum, and 24 or 36 give
acceptable accuracy for all the reactions tried so far.


MAXL,MINL are the maximum and minimum $L$ values
for the non-local
kernels. If zero, MAXL has the default value JTMAX+6, and
if MINL $<$ 0 it takes the default value $|$JTMIN$|$-6.


MTMIN is the lowest L-transfer for calculating transfer form factors
using the m-dependent expressions for spherical harmonics.
Putting MTMIN = 0 gives default value MTMIN = 6 (use MTMIN $<$ 0 to avoid
invoking default, if all transfers are to use this method).

EPC = percentage cutoff accuracy in the NNU angular integration.
If zero, the default is (30/NNU)$^2$\%.

\item[ERANGE, DK] set default parameters for continuum bins:

ERANGE = range of energies of the upper and lower boundaries
of continuum bins:\\
if ERANGE $>$ 0, then {\em ratio} of these energies;\\
if ERANGE $<$ 0, then {\em difference} of the energies in MeV.

DK     = step size of $k$ (fm$^{-1}$) for integration over the ERANGE to construct the continuum bin.

 \item[PLANE, ELPMAX] ~\\
PLANE = 1,3: zero Coulomb potential for elastic channel\\
      = 2,3: zero Coulomb potentials for all nonelastic channels.

ELPMAX = energy ELAB below which the LPMAX limits in each partition are applied 

\item[RELA, RELREF]
$\neq$\upq \upq: use relativistic kinematics for the incident projectile:\\
If RELA contains `a', use Ingemarsson eq(16) for kinematics\\
If RELA contains `b', use Ingemarsson eq(17) for kinematics\\
If RELA contains `h', use kinematics of G. Hale (LANL) with RELREF as the reference partition\\
If RELA contains `f', use Ingemarsson eq(19) to scale potentials\\
If RELA contains `g', use Ingemarsson eq(21) to scale potentials\\
If RELA contains `r', get reduced mass from ratios of total energies

\item[COMPLEXBINS, CCBINS, SUMFORM, OMPFORM]  ~\\
If COMPLEXBINS=T, allow for complex-valued bins (e.g. from optical potentials)\\
If CCBINS=T, allow for coupled-channels bins, which are necessarily complex-valued.\\
If SUMFORM = 1, 2 or 3: different methods for sum reductions of core-excitation bins. \\
OMPFORM  = 0, 1, 2: different methods of parallel calculation of core-excitation bins.


\item[PLUTO(:)]
 =  list of potential numbers KP to be prepared for the Lagrange-mesh method for single-particle eigenstates. 

\item[UNITMASS] : unit (in amu) for MASS values read in.
Default = 1.000

\item[FINEC]: 1/(fine-structure constant): used to determine electrostatic $e^2$.
Default = 137.03599.

\end{description}

\subsubsection{Incident channel}
\begin{description}
\item[PEL, EXL, LAB, LIN, LEX]
Incoming plane waves are present in partition PEL with excitation
pair EXL.  The energies ELAB(:) are the laboratory energies for
partition LAB's nucleus LIN (1 or 2 for projectile or target)
in excitation pair LEX.

The defaults for PEL,EXL,LIN \& LEX are all 1, and the default
for LAB is PEL, so these variables can be normally omitted.

\item[ELAB(:5), NLAB(:4)] :
Solve at different laboratory energies $E$ until ELAB(i) = 0 is found.\\
 If NLAB(i) $>$ 1, then the range of $E$ from ELAB(i) to ELAB(i+1) is covered in NLAB(i) linear intervals.\\
 For just one energy $E$, just specify 'ELAB = $E$'.

\end{description}


\subsubsection{Solving equations}

\begin{description}
\item[IPS, IT0, ITER, FATAL, IBLOCK, PADE]

Solve the coupled channels equations by at least IT0 iterations, and
up to ITER iterations.
Stop sooner if the absolute differences between successive $S$-matrix
elements (scaled by (2$J$+1)/(2.JTMAX+1)) are less than IPS percent.
(Excited state pairs with IGNORE set in the \&STATE namelist are not counted against IPS).

Putting IT0=ITER zero solves only the elastic channel (along with the
IBLOCK channels: see below).
Putting IT0=ITER = 1 or 2 etc. gives 1 or 2-step DWBA.

Normally, a run is terminated if more than ITER steps are required for
convergence. \\
Setting FATAL=False allows continuation even after
convergence has failed after abs(ITER) iterations.

Iterations are normally also stopped if the successive differences are smaller
than the errors estimated for the numerical integration of the coupled
equations. Setting IPS $< 0$ uses abs(IPS), without this extra check.

IBLOCK is the number of pairs of excitation levels
(starting from partition 1, excitation 1) that are coupled exactly by
blocking together.
%If IBLOCK$<$1, then read Line 4.5 (see below) for R-matrix solution of the
%coupled equations.
%\\

PADE
 = 0 for no Pade acceleration,
\\ = 1 for Pade acceleration by the epsilon algorithm,
\\ = 2 for Pade acceleration by finding the N/D polynomials.

\item[NOSOL] : if not to solve the CRC equations, only construct couplings

\item[DRY]
normally F (false), but if T (true) the code does a `dry run'
to check that all arrays are of sufficient size.
All coupled channel sets are generated, but only the elastic
channels should be non-zero.

 \item[SMALLCHAN, SMALLCOUP, CCREAL] ~\\
%
SMALLCHAN = fraction of unitarity to define a 'small channel'. 
A channel that is `small' for NSMALL=5 times is dropped permanently.

SMALLCOUP: if all nonelastic channels are weaker than the fraction
SMALLCOUP of unitarity, then permanently change from coupled-channels to DWBA.

CCREAL = T: assume all potentials and couplings real-valued

\item[PSIREN]: do simple renormalisation of channel wfs after Pade acceleration
\item[INITWF]: read in external scattering wave functions before iterating couplings.
$> 0$: read in formatted data from file number INITWF.\\
$< 0$: read in unformatted data from file number INITWF.\\
The file format is that which is produced when WAVES$>0$. 
All channels not read in are set to zero.

\item[MPIHELP]: number of OPENMP threads allocated to each MPI node

\item[MAXCOUP(:), EXPAND(:)]: fudge limits and factors (respectively) to adjust sizes of generated work arrays.

\end{description}


\subsubsection{R-matrix setup}
\begin{description}

\item[NRBASES] ~\\
When NRBASES$\neq$0, R-matrix solutions are selected then all channels
are `blocked' together and solved in a full CRC procedure. All non-local potentials are included to all all orders (not iteratively).

NRBASES = target number of radial basis states in each channel.
(Use 2*NRBASES for the elastic and first-inelastic channel for more accuracy).\\
If NRBASES $<$ 0, then use {\em Lagrange mesh} basis with NLAG=--NRBASES basis functions.

\item[NLAGCC] $>0$: another way to set NLAG = NLAGCC for a Lagrange mesh calculation.

\item[NRBMIN, BUTTLE, PRALPHA, PCON, RMATR, BNDX(2), BTYPE, WEAK] ~\\
NRBMIN = minimum number of radial basis states (default NRBASES).

BUTTLE = 4 for none, 0,2 for complex, 1,3 for real
(2,3 without energy shift) Buttle correction.\\
(default 0)

PRALPHA = print basis-state eigenvalues to files fort.60,61,62,63

PCON = trace variable for calculation of radial basis states (same
meaning as IPC for single-particle bound states).


RMATR = R-matrix matching radius (default RMATCH ).
Warning: RMATR will be changed to an even multiple of HCM.

BNDX(i) is used to set the R-matrix Boundary Condition number depending on BTYPE.
Use i=1,2 for positive, negative parity coupled channels sets (respectively).

Energy $\hbar^2k^2/2m$, where $k= f'/f$, the logarithmic derivative
for all radial basis states at $r$=RMATR, with $k$ having the same
sign as EBETA(i). 

BTYPE = E,k, L,B, S : methods to calculate Bloch parameter B from BNDX.
If BNDX(i)  is the energy for 'E' or  wave number for 'k, then B is inverse logarithmic derivative of Coulomb function there. Otherwise B = --L for 'L', B = BNDX(i) for 'B', and S (shift function) for 'S'.

If WEAK$>$0, then non-elastic columns of the R-matrix are set to zero,
  when penetrabilities $<$ WEAK.
  
\item[KRM]
Change defined coupling order of R-matrix parameters. Best use default KRM=0.
 
\item[EIGENS, MEIGS, VSEARCH, ECHAN, ENODES] : if EIGENS, then: \\
MEIGS = maximum number of bound states to find, \\
VSEARCH =  location of potential form to adjust so that channel ECHAN has ENODES nodes.

\item[NPARAMETERS] Read in this number of R-matrix parameter namelists after the \&FRESCO namelist.

\item[EOBS] : if EOBS=T, treat input R-matrix widths as `observed' values. Otherwise treat at 'formal' widths.
\end{description}

\subsubsection{Trace variables}
A value of 0 gives no trace,
increasing values give progressively more printed output.\\
{\em Decremented} variables are decreased by 1 on each use.

\begin{description}
\item[CHANS] $\geq$ 1 : Print the sets of coupled partial waves for each J,parity.
Decremented.

\item[LISTCC] = 1 : Print coupling coefficients between these channels. Decremented.
\\  = 1,2,... Print progressively more detail of couplings.
\item[TRENEG] $\geq$ 1 : Print all multipole potentials
\\  $\geq$ 3 : Print all monopole potentials as well
\item[CDETR]  $\geq$ 1 : Print information on the solving of the coupled equations.
(decremented).
\item[SMATS]  $\geq$ 1 : Print absorbtion \& reaction cross sections for successive
partitions and excitations.
\\  $\geq$ 2 : Print elastic S-matrix elements ($S_{el}$).
Also `punch' these elastic $S_{el}$ on output file 7,
in format (2F15.10,L,J,JTOT) for $S_{el}$ complex, L, J and JTOT.
See WDISK below for description of these quantum numbers.
\\  $\geq$ 3 : Print all S-matrix elements for the `grazing partial waves'
defined by $0.05 < Re(S_{el}) < 0.95$
\\  $\geq$ 4 : Always print all the S-matrix elements.
\\  $\geq$ 5 : Print all S-matrix elements at each iteration of the coupled
equations (or, if PADE  $>$ 0, the Pade approximant)
\\  $\geq$ 6 : Print all actual S-matrix elements at each iteration (these
may be divergent before Pade acceleration).


\end{description}

\subsubsection{Output details}

\begin{description}
\item[XSTABL, NLPL, WAVES, LAMPL, VEFF, KFUS, NFUS, WDISK, BPM, MELFIL, CDCC,  TCFILE, tcfilename]

\item[XSTABL] $\neq$ 0 : If XSTABL is non-zero, in file 16 punch output cross sections
for all excitation levels in all partitions.
A header line in FORMAT(5I6) gives partition IC, level pair IA,
number of tensor ranks of analyzing powers 1 $<$ KQ1PR $<$ XSTABL,
number of angles NANGL, and NEARF.  NEARF=1 for total cross section,
2 for far-side component, and 3 for near-side component.

Then follow NANGL print operations in
{\tt FORMAT(1P,6E12.4)}, repeating
the {\tt FORMAT} for each operation if KQ1PR is large,
of THETA, elastic xs (mb), $T_{10}$, $iT_{11}$,
$T_{20}$, $ T_{21}$, $ T_{22}$, $ iT_{30}$, $ iT_{31}$
etc.

\item[NLPL]   $>$ 0 : print a `contour plot' of the non-local kernels
$K_{fi}(R_{f},D_{fi})$.  This is useful to determine if the parameters in the \&Fresco namelist
are adequate.  Decremented.

\item[WAVES]
  $\pm$1 or $\pm$3 : print out wave function solutions of the coupled
equations at the end of the iterations.
(If WAVES$<$0 : print out the RATIO of the w/f to its asymptotic
form $((G-iF) - S.(G+iF)).i/2$)
\\  2 or 3     : print out the source terms at
{\em each} iteration of the coupled equations.



\item[LAMPL]
  $\neq$ 0: Print out (on Fortran file 36)
the coefficients $A(m'M':mM; L)$ for the Legendre
coefficients in the scattering amplitude for the partition number
abs(LAMPL),
{\em and}
print out the $f(m'M':mM; \theta)$ for each angle $\theta$.
\\ $<$0 : only print out on file 37 the amplitudes $f$, not the $A$'s,
for partition abs(LAMPL).

The phase convention here is that there is no Coulomb phase shift
for $L = 0$ in the Coulomb scattering amplitude : factors such as
$\exp i(\sigma_L-\sigma_0))$  appear in the $A$'s.


\item[VEFF]
  $\neq$ 0 : Calculate the `coupled channels effective potential'
found be averaging the `trivially equivalent potential' over all
the $J,\pi$ sets, with weights of the elastic wave functions
times the reaction cross section, in each set.
\\  $<$  0 : Add this effective potential to optical potential
of the elastic channel before printing.
\\  = --2 or +2 : Exclude partial waves with elastic $S$-matrix element
$S_\ell< 0.1$ from the averaging sum.
\\ The results show the real and imaginary parts for successive
values of $J-L$, for the projectile only.

\item[KFUS, NFUS]
If KFUS  $>$ 0 :  Calculate `core fusion' using the imaginary and scalar
parts of potential number KFUS (i.e.\&pot namelists with TYPE = 1 or 2,
and KP = KFUS), also for the first NFUS inelastic channels.



\item[WDISK]
  = 1 :  Print elastic wave functions on output file 17, FORMATTED
\\  = 2 :  Print   all   wave functions on output file 17, FORMATTED
\\  =--1 :  Print elastic wave functions on output file 17, UNFORMATTED
\\  =--2 :  Print   all   wave functions on output file 17, UNFORMATTED


\begin{quotation}
\noindent
The following data formats are used when WDISK $>$ 0 :
\begin{verbatim}
line A: (I4,2F8.4,F8.1,I3)
         NR,H,ENLAB,JTOTAL,PARITY,MP,MT,ZP,ZT :
           number radial points, step size, lab. energy,  J,pi,
           projectile and target masses and charges
line B: (2I4,2F6.1,I4,F6.1,2F15.10,f12.8)
         IT,L,J,JTOT,LIN,JIN,SMAT (complex), ETA
\end{verbatim}
where

IT = index to excited state pair, counts lines 7.\\
L  = partial wave\\
J  = L + projectile spin\\
JTOT = total spin = J$_{total}$ = J + target spin\\
LIN  = incoming partial wave\\
JIN  = incoming J value.\\
SMAT = S matrix element for this partial wave.\\

\begin{verbatim}
line C: (6E12.4)   (psi(I),I=1,NR)   wave function
line C is repeated until NR complex values given
NB: the first point psi(1) = 0 always, as at r=0
\end{verbatim}
Lines B \& C are repeated for each channel, until IT $<$ 0.
\\
When WDISK $<$ 0, successive records contain the two real values
of psi(I), starting IN THIS CASE, from I=2 (i.e. $r=h$).
\end{quotation}

\item[BPM]
  $\geq$ 1 : Calculate fusion cross sections in the Barrier Penetration
Model using first the bare potential, and then the bare potential + the
`weighted equivalent potential' calculated when VEFF $\neq$ 0.
\\  $\geq$ 2 : Print out $L$-distributions of the fusion cross section.

\item[MELFIL]
  $\ne$ 0 : Write real files 53 and 54 in the `mel' and `spec' format
        for use e.g. by R-matrix program {\tt sturmxx}.
\\  = $\pm$2 : Write complex files 53 and 54 in the `mel' and `spec' format.
\\   $<$ 0 : Write file 53 `mel' in text format format.

\item[CDCC = 1]:
Print out the $f(m'M':mM; \theta)$ for each angle $\theta$
on file 57 for partition {\tt PEL}, after the following information for {\em uncoupled} bin states:
{\footnotesize
\begin{verbatim}
line Y:(i2)  1    (indicating CDCC=1 format below)
line Z: (A120)   HEADNG from Fresco input.
line A: (F10.4,3F8.4)   ENLAB,Bproj,H2SM,e^2,Btarg,inp, (Qval if inp=1)
              lab energy,projectile binding energy, hbar^2/2.m, e^2, 
              target binding energy,inp, Qval if inp=1
                                                
line B: (7f8.4)         massp,masst,massc,massv,massr  
                                         masses: projectile,target,core,valence,residual
line C: (7f8.4)         Zp,Zt,Zc,Zv,Zr                charges
line D: (7A8)           namep,namet,namec,namev,namer names
line E: (7f8.1)         Jp,Jt,Jc,Jv,Jr                g.s. spins
line F: (7i8)           Pp,Pt,Pc,Pv,Pr                g.s. parities
   If inp=1, cards B-F (incl) have further #6 and #7 values 
                for 'initial projectile' and 'initial target' too.
   
line G: (4I4)           NBINS,NKMAX,NEXB,NNJMAX no. CDCC bins, max NK,
                                                no. excited states, max(2*Jex+1)
line H: (I4,2f8.4)      NANGL,THMIN,THINC       (cm angular range from \&FRESCO)
for each of the NBINS bins:
 line I:(i2,2f4.1,3f8.4,2i4)
         l,j,Emid,kmin,kmax,NK,KN,ISC
            l,j: quantum numbers  (s==Jv)
            Emid:  centre of bin with respect to continuum threshold
            kmin,kmax,NK: Min,max and number of k values in bin integral
            KN:  original KN index for bin state
            ISC:  normalisation used for bin
     for each IK=1,NK
         line J: (10f8.4) delta(IK): nuclear phase shift used in bin integral (radians)

for each excited state pair in the entrance partition: IA=1,NEXB::
    line K: (f4.1,i4,f8.4,i4) Jex,Parity,Eex,IBIN:
            Jex :          spin of this projectile excited state  (not including core spin)
            Parity:        parity of this projectile state
            Eex:           excitation energy of this state above g.s.
            IBIN:          (first) bin defined for this excited state
    for each IANG=1,NANGL: read complex numbers:
    line L: (6E12.4): ((FAM(MEX,MP,IANG,IA),MEX=1,2*Jex(IA)+1),MP=1,2*Jp+1)
\end{verbatim}
}

The phase convention for all CDCC values is that there is no Coulomb phase shift
for $L = 0$ in the Coulomb scattering amplitude : factors such as
$\exp i(\sigma_L-\sigma_0))$  appear in the $A$'s.t

~\\
Summary of bin normalisation factors for different ISC values:

ISC\\
\\ = 2:  $\exp(-i\delta(k))$
\\ = 4:  $\sin(\delta(k))\exp(-i\delta(k))$
\\ = 12:  $k\exp(-i\delta(k))$
\\ = 14:  $k\sin(\delta(k))\exp(-i\delta(k))$

\item[CDCC = 2]:
Print out the $f(m'M':mM; \theta)$ for each angle $\theta$
on file 57 for partition {\tt PEL}, after the following information for {\em coupled bin states}:


{\footnotesize
\begin{verbatim}
line Y:(i2)  2    (indicating CDCC=2 format below)
line Z: (A120)   HEADNG from Fresco input.
line A: (F10.4,3F8.4)   ENLAB,Bproj,H2SM,e^2,Btarg,inp, Qval if inp=1
                  lab energy,projectile binding energy, hbar^2/2.m, e^2, 
                  target binding energy,inp, Qval 
line B: (7f8.4)         massp,masst,massc,massv,massr  
                                         masses: projectile,target,core,valence,residual
line C: (7f8.4)         Zp,Zt,Zc,Zv,Zr                charges
line D: (7A8)           namep,namet,namec,namev,namer names
line E: (7f8.1)         Jp,Jt,Jc,Jv,Jr                g.s. spins
line F: (7i8)           Pp,Pt,Pc,Pv,Pr                g.s. parities
   If inp=1, cards B-F (incl) have further #6 and #7 values 
                for 'initial projectile' and 'initial target' too.

line G: (5I4)           NBINS,NKMAX,NEXB,NNJMAX,NCHMAX
                                                no. CDCC bins, max NK,
                                                no. excited states, max(2*Jex+1), max nch
line H: (I4,2f8.4)      NANGL,THMIN,THINC       (cm angular range from \&FRESCO)
line I: (I4)                NCE              (number of excited states. NCE=0 for only gs).
for each of the ICE=1...NCE core excited states  (card skipped if NCE=0)
lines J: (I4,2f8.4)      IPARCE, JCE, ECE    (parity -1,+1; spin; energy of excited states)
                         (for the gs: parity=Pc, spin=Jc, energy=0.0)
for each of the NBINS bins:
 card K:(f4.1,2I4,f4.1,3f8.4,2i4)
         Jex,Pex,nch,Emid,kmin,kmax,NK,KN,ISC
            Jex: overall spin  [ (l s)j, JCE(ICE); Jex>
            Pex: overall parity = parity(ICE) * (-1)**l
            nch: number of partial wave channels coupled to Jex/Pex.
            IL: incident channel (1<= IL <= nch)
            Emid:  centre of bin with respect to continuum threshold
            kmin,kmax,NK: Min,max and number of k values in bin integral
            KN:  original KN index for bin state
            ISC:  normalisation used for bin
         For each partial wave c=1..nch
         line L: (i4,f4.1,i4)
            l,j,ICE: quantum numbers. Use s=Jv.            
     for each IK=1,NK for    k=kmin+(IK-1)*kinc where kinc = (kmax-kmin)/(NK-1)
         line M: (2f10.6) delta(IK),k: any nuclear phase shift used in bin integral (radians)
         line N: (10f10.6) S(:,:)     the full scattering S matrix for nch channels 
                                        (always nch*nch, so closed channels included)

for each excited state pair in the entrance partition: IA=1,NEXB::
    line K: (f4.1,i4,f8.4,i4) Jex,Parity,Eex,IBIN:
            Jex :          spin of this projectile excited state (including core spin)
            Pex:           parity of this projectile state
            Eex:           excitation energy of this state above g.s. (including core energy)
            IBIN:          (first) bin defined for this excited state
    for each IANG=1,NANGL: read complex numbers:
    line L: (6E12.4): ((FAM(MEX,MP,IANG,IA),MEX=1,2*Jex(IA)+1),MP=1,2*Jp+1)
\end{verbatim}
}

\item[TCFILENAME]: 
File to write HF data, default fort.5420.

\item[TCFILE]: 
Print in file TCFILENAME data needed for Hauser-Feshbach code YAHFC. These are penetration probabilities and  all angular distributions.

\item[DIST0] :
If DIST0 $\geq 0$, cut off charged-particle elastic scattering at below DIST0 degrees, for making GNDS or ENDF evaluation files. Default --1.

\item[TMP]:  name of directory for temporary files: `/tmp' or `.'  Default = /tmp/

\item[GRACE, BOLDPLOT] : for use after SFRESCOX searches.\\
If GRACE=T, print plotting metadata for xmgrace, not xmgr. Default=T.\\
BOLDPLOT: make width=2 lines in plots, not width=1, and all in monochrome. More suitable for publications.

\end{description}
%\newpage


\subsection{\&PARTITION and \&STATES namelists}
%
The \&PARTITION namelists are repeated until an empty \&PARTITION namelist is read in.
After each non-empty (NEX $> 0$) \&PARTITION, NEX instances of the \&STATES namelist are required.


\subsubsection{\&PARTITION namelist}
%
\begin{description}
\item[NAMEP,  MASSP,  ZP,  and NAMET,   MASST,    ZT] ~\\
NAMEP = name of projectile-like nuclide (string 8 characters max),  \\
MASSP = mass of projectile (in units of UNITMASS), \\
ZP = charge of nuclide, \\
NAMET, MASST, ZT = equivalent properties of target-like nuclide.

\item[NEX,  QVAL, PWF] ~\\
abs(NEX) = number of excited-state pairs in this partition. \\
If NEX $<$ 0, then cross sections for this partition are not printed\\
QVAL = $Q$-value (in MeV) of this partition for Ep+Et=0.0. Only differences in $Q$-values between partitions are significant.\\
PWF = use asymptotic piecewise methods in this partition (default is True).

\item[READSTATES, PRMAX, LPMAX, MIXPOT]  ~\\
READSTATES = read in specification of the NEX states from file number READSTATES. \\
PRMAX = maximum radius in this partition. Useful for R-matrix calculations, but redefines what `local' interactions mean.\\
LPMAX = maximum partial wave $L$ in this partition if the incident lab energy is below ELPMAX (given in the \&Fresco namelist). Default=$-1$ (no limit),

\item[MIXPOT] governs the couplings that are generated (in this partition) between levels that have {\em different} KP potential indices.
It does not affect any couplings when the KP index values are identical.

 $\leq$ 0: {\em no} couplings (default)
\\ = 1:  couplings only to or from the first state in that partition
\\ = 2:  couplings only according to the KP value of the destination level
\\ = 3:  {\em all} couplings produced by either KP value.


\end{description}


\subsubsection{\&STATES namelists}
Repeated IA = 1,..,NEX times before the next \&PARTITION: 

\begin{description}

\item[Jp,  COPYp,  BANDp , Ep,  KKp, Tp, ~~ Jt,  COPYt, BANDt , Et,  KKt, Tt] ~\\
For both projectile (p) and target (t), the following are read:

J     = spin of the state (Jp and Jt, and similarly below).

BAND  $>$ 0 for positive parity state, and $<$ 0 for negative parity.
The value of BAND defines the rotational band : all states of the
same BAND may be coupled by rotational mechanisms.
Odd multipoles couple BAND values of opposite sign.

If COPY is positive, then this level is a copy of a
previous level number (COPY $<$ IA) in the same partition.

If COPYp $<$ 0, then this level is an exchange copy of
{\em the same IA numbered}
level in the previous partition abs(COPY).
That is, a copy with projectile and target nuclei exchanged.

If COPYt = -IC, then the target state is an exchanged projectile.
If that state has spin zero, then only even partial waves are allowed.

COPYs are needed to define identical states of one nucleus in different
pairings IA.   Spectroscopic amplitudes (in \&cfp namelists) should be defined
for only the original `copied' level.

E     = excitation energy of state relative to ground state.

KK     = K of rotation band containing this state.
(If zero, use spin of first level in this band.)

T     = isospin of this state (if zero, use INT((mass-2*charge)/2))

\item[KP, FEXCH,IGNORE, INFAM,OUTFAM] ~\\
KP = index of optical potential for this pair of excited states.
If zero, use as default KP = IC, the number of this partition.

If FEXCH = T calculate and write out all cross sections and amplitudes
for $180-\theta$.


If IGNORE = T, ignore the convergence of this  state pair
(see IPS discussion above).

If INFAM $\ne$ 0,  read in complex amplitudes later to add to
calculated scattering amplitudes\\
$<$ 0, read in \underline{only one} amplitude (non spin-flip)
for each scattering angle, with others zero.

If OUTFAM $\ne$ 0,  write out complex amplitudes later to add to
calculated scattering amplitudes.\\
$<$ 0: write out \underline{only one} amplitude 
for each  angle, namely the average of diagonal amplitudes.


The output files are suitable to be reread later as input file
\end{description}

\newpage
\subsection{\&POT potential namelists}
%
Each potential is started by a \&pot namelist with TYPE=0, defining the radii to be used
and the diagonal Coulomb potential, followed by any number of further \&pot namelists with TYPE$>$0,
which cumulatively construct the components of the nuclear potential.
The KP value in each namelist identifies the potential of which this is to be a component.
The \&pot namelists are repeated until KP$<0$ indicates the last one to be considered, or an empty namelist is read in.


\subsubsection{First kind of \&pot namelist}

\begin{description}
\item[KP, TYPE,  p1, p2, p3, p4]  with TYPE=0, or 
\item[KP, TYPE,  AT, AP,  RC, AC] with TYPE=0 : \\
%
    All the radii are multiplied by CC. \\
       CC is updated by a TYPE=0 \&pot namelist to CC = AT$^{1/3}$ + AP$^{1/3}$ where AT$\equiv$p1 and AP$\equiv$p2, \\
       CC is updated by TYPE$>$0 namelists, if P0 $>$ 0,  to CC = P0$^{1/3}$.

RC$\equiv$p3 = surface radius (when multiplied by CC) of charge distribution\\
AC$\equiv$p4  = diffuseness of charge distribution



The values of AT and AP affect {\em only} the conversion from $r_0$ radii
to R radii in the definition of potentials in these namelists.
If however you adopt the convention that AT is the target mass number
and AP the projectile's, then FRESCOX can later display the
fractional $\beta$ values corresponding to given deformation lengths.
(These are for information only, and do not affect the couplings.)


\end{description}
\subsubsection{Second kind of \&pot namelist: TYPE$>$0}

\begin{description}
\item[KP, TYPE,  IT, SHAPE, p1, p2, p3, p4, p5, p6, p0] or 
\item[KP, TYPE,  IT, SHAPE, p(:)] or 
\item[KP, TYPE, IT, SHAPE,  V,Vr0,Va,   W,Wr0,Wa] 
\item[{\rm also}: LSHAPE, JL, XLVARY, ALVARY, DATAFILE]
\end{description}

An empty namelist  ends reading potential parameters, as does a negative KP value in the last namelist.
%If the namelist is empty or KP $\leq$ 0, this will be the last \&pot namelist. 
%If KP $<$ 0, use abs(KP) below, and omit the empty namelist.
%%
\begin{tabbing}
\hspace*{5mm} \= \hspace*{1cm} \= \hspace*{2cm} \= \\
For\>TYPE\\
\>0\> \>Coulomb potential (see above)\\
\>1\>SHAPE\>Central potential, Volume\\
\>2\>SHAPE\>Central potential, Derivative (i.e. surface)\\
\>3\>SHAPE\>Spin-orbit for projectile\\
\>4\>SHAPE\>Spin-orbit for target\\
\>5\>SHAPE\>Tr tensor force for projectile\\
\>6\>SHAPE\>Tr tensor force for target\\
\>7\>SHAPE\>Tensor force between L and combined projectile+target spins
(e.g.~n-p tensor force)\\
\>8\>SHAPE\>Spin.spin force for target \& projectile spins\\
\>9\>SHAPE\>Effective mass: reduction from unity\\
\>10\>SHAPE\>Deformed projectile   (matrix elements from ROTOR)\\
\>11\>SHAPE\>Deformed target       (matrix elements from ROTOR)\\
\>12\>SHAPE\>Projectile coupled by matrix elements read in.\\
\>13\>SHAPE\>Target     coupled by matrix elements read in.\\
\>14\>SHAPE\>Projectile second-order coupled by matrix elements read in.\\
\>15\>SHAPE\>Target second-order coupled by matrix elements read in.\\
\>16\>SHAPE\>Target \& projectile simultaneous second-order coupled by matrix elements read in.\\
\>17\>     \>Target \& projectile all-order coupled by matrix elements read in.\\
\>20\>NUM\>Super-soft N-N potl. of de Tourreil \& Sprung (SSC(C))\\
\>21\>NUM\>User-supplied N-N potential via subroutine NNPOT.
\end{tabbing}
%

\noindent
TYPE indicates type of spin tensor or excitation coupling,

\noindent
IT is 1 or 3 to make this component included only iteratively,\\
 \hspace*{8mm} 2 or 3 to NOT subtract this component in KIND=3,4 single-particle  couplings,

\noindent
SHAPE indicates the radial shape of the form factor,

\noindent
P1-P3 \& P4-P6  are parameters for the real and imaginary parts (respectively).

\noindent
P0$\equiv$P7 can give a further parameter.

\medskip

If TYPE $<$ 0, add new potential numerically into previous potential,
and use abs(TYPE) instead.   This saves space, and is useful if adding
 components to be deformed by a following namelist with $\leq$ TYPE $\leq$ 13.


\bigskip

The SHAPEs for volume potentials (TYPE=1, 8 \& 15) are,
\\
with R = P2 * CC, RH = (r - R)/P3, and E = exp(-(r-R)/P3),
\\
(replacing (P1,P2,P3) by (P4,P5,P6) throughout for the imaginary parts)\\
%
   0: Woods-Saxon  = - P1 / (1 + 1/E)
\\ 1: WS squared   = - P1 / (1 + 1/E)$^2$.
\\ 2: Gaussian     = - P1 * exp(-RH$^2$.)
\\ 3: Yukawa       = - P1 * E / r
\\ 4: Exponential  = - P1 * E
\\ 5: Reid soft core for $T$=0, central part
\\ 6: Reid soft core for $T$=1, central part
\\ 7: Read Real
\\ 8: Read Imaginary
\\ 9: Read Complex
\\--1: Fourier-Bessel = $j_0$(RH) = $\sin$(RH)/RH


For SHAPEs 7-9, immediately read in free format from from {\em Input File 4}, or from {\bf datafile} if specified:
First a comment line,\\
then (free format) NPOINTS, RSTEP, RFIRST, \\
followed by NPOINTS potential points for the shape in steps of RSTEP starting at r=RFIRST.
\\
Rescale the real part by P1, the imaginary part by P2,
and use radius = P3 * CC for subsequent deformation calculations.
\\
If SHAPE = -7, -8, or -9, rewind file 4 or {\bf datafile} before reading from it.\\
If {\bf datafile} = {\tt continue}, then continue reading the previous file (do not rewind or reopen).\\
\bigskip

The SHAPEs for surface potentials (TYPE=2), with R, RH and E as before,
are the first derivative forms (normalised to --1 when E=1,
if reasonable):

\noindent
      0 : Woods-Saxon  = - P1 * 4 * E / (1+E)$^2$ \\
      1 : WS squared   = - P1 * 8 * E*E / (1+E)$^3$\\
      2 : Gaussian     = - P1 * 2 * exp(-RH*RH) * RH\\
      3 : Yukawa       = - P1 * E * (1 + r/P3) / (r*r)\\
      4 : Exponential  = - P1 * E\\
      5 : Reid soft core for $T$=0, spin-orbit part $\times r$,\\
      6 : Reid soft core for $T$=1, spin-orbit part $\times r$,\\
      7,8,9 : Read from external file, as above.
\bigskip

The SHAPEs for spin-orbit potentials (TYPE=3 or 4) are the following surface
shapes, \\ where CONLS = $\hbar^2/(m_\pi^2 c^2)$ = 2.000.

\noindent
      0 : Woods-Saxon  = -- CONLS * P1 *  E / [(1+E)$^2$ * P3 * r] \\
      1 : WS squared   = -- CONLS * P1 * E*E / (1+E)$^3$  / (2*r*P3) \\
      2 : Gaussian     = -- CONLS * P1 * exp(-RH*RH) * RH  / (2*r*P3) \\
      3 : Yukawa       = -- CONLS * P1 * E * (1 + r/P3) / (4*r$^3$*P3)\\
      4 : Exponential  = -- CONLS * P1 * E / (4*r*P3)\\
      5 : Reid soft core for $T$=0, spin-orbit part\\
      6 : Reid soft core for $T$=1, spin-orbit part\\
      7,8,9 : Read from external file, as above.

To become interaction potentials,
these form factors will be multiplied by
$j(j+1) - l(l+1) - s(s+1)= 2{\bf l}.{\bf s}$, where $s$ = spin of projectile or target
in each of its ground state and excited levels. There is no option here for
state-dependent spin-orbit forces.
\bigskip

The SHAPEs for tensor potentials (TYPE=5, 6 \& 7), with R, RH and E as before,
are the second derivative forms (normalised to unity when E=1,
if reasonable):

\noindent
      0 : Woods-Saxon  = P1 * 8 * E*(E-1) / (1+E)$^3$\\
      1 : WS squared   = P1 * 4 * E*E / (1+E)$^3$\\
      2 : Gaussian     = P1 * 2 * exp(-RH*RH) * RH\\
      3 : Yukawa       = P1 * E * (1/P3$^2$. + 2/r$^2$.) / r\\
      4 : Exponential  = P1 * E\\
      5 : Reid soft core, 12 * tensor part\\
      6 : Vol Gaussian = 12 * P1 * exp(-RH*RH)\\
      7,8,9 : Read from external file, as above.

\medskip

If SHAPE $\geq$ 30, then use SHAPE-30, and define a
$J$- or  $L$-dependent potential by a factor defined
using \\
{\bf JL, LSHAPE, XLVARY, ALVARY} \\
so that the potential is multiplied by a form factor depending LSHAPE as 

\noindent
 =0 : Woods-Saxon  = 1 / (1 + 1/E), \\
      =1 : WS squared   = 1 / (1 + 1/E)$^2$\\
      =2 : Gaussian     = exp(-RH$^2$), \\
where X=L (for JL=`L'), X=J (JL=`J'),
RH = (X-XLVARY)/ALVARY, and E=exp(-RH).

\medskip

If SHAPE $\geq$ 10 and $<$ 19, then use SHAPE-10, and write out the
resulting potential to {\em Output file 25} in FORMAT(6E12.4):
a form suitable for subsequent reading by a SHAPE $\geq$ 20.

\medskip

If SHAPE $\geq$ 20 and $<$ 30, then a $J_{T}/\pi$-dependent
potential is defined.
\\
For each subsequent coupled-channels set with total angular momentum
$J_{T}$, a form factor is read into this potential from
{\em Input file \# SHAPE}, reading M = RMATCH/HCM+1 complex numbers
in FORMAT(6E12.4) from the
file as a random-access, with (M-1)/3+1 lines for each form factor.
\\
For each CC set, form factor number
$int(J)+1$ (for SHAPE $\geq$ 24) or
1/2 for +/- parity (for SHAPE $<$ 24) is read.

If SHAPE = 40, then a $\pi$-dependent potential is defined:
use KP = P1 for + parity CC sets, and P2 for -- parity.

If SHAPE = 41, then a $L$-dependent potential is defined:
use KP = P($L$+1) for L=0--5, and P0 for L$\geq$6.

If SHAPE = 42, then a $J$-dependent potential is defined:
use KP = P(Ji+1) for Ji=0--5, and P0 for Ji$\geq$6, with Ji = int($J$).

If SHAPE = 43, then a $L$-parity-dependent potential is defined:
use KP = P1 for $L$ even, and P2 for $L$ odd.
\\
There need only be one \&pot namelist for 40 $\leq$ SHAPE$\leq$43. The values of TYPE
and P2-P7 are ignored. No files are used. Choices are listed in fort.48

If SHAPE = 45, then an energy-dependent potential is defined, using linear interpolation of the potential forms.
The nodes are KP=P1 at energy P2, P3 at energy P4, etc., up to KP=P9 at energy P10 (MeV).


\bigskip
{\em Nucleon - nucleon Potentials}
\bigskip

The SHAPEs for the nucleon-nucleon SSC (C) potential (TYPE = 8) are
taken from the subroutine SSCC in FRESCOX.
The input NUM is the number of angular momentum components stored,
using the order
\\
 $^1S_{0}$, $^3S_{1}$, $^3S-^3D$, $^3D_{1}$, $^1P_{1}$, $^3P_{0}$
$^3P_{1}$, $^3P_2$, $^3P-^3F$, $^3F_2$, $^1D_2$, and $^3D_2$
\\
Note that TYPEs 8 \& 9 can only be used in KIND = 1 ($LS$-coupled)
single-particle states.


The SHAPEs for the user's nucleon-nucleon potential (TYPE = 9) are
taken from the subroutine NNPOT, with NUM the same meaning as above.

\subsubsection{Deformations}


The SHAPEs for deformed potentials (TYPE $\geq$ 10) are constructed either
by reading external forms (SHAPE = 7, 8 or 9) for each P(k) non-zero
(the input k-form being multiplied by P(k)),
or (for SHAPE $\geq$ 10) by deforming the potential on the
{\em immediately preceding} input namelist.
This preceding namelist must be of TYPE=0 for Coulomb deformations,
or TYPE = 1 or 2 for nuclear deformations: deformation of non-central
forces is not yet implemented.
\bigskip

For Nuclear deformations P(k) = DEF(k), the deformation lengths (in fm.).
(P0 = DEF(0) for {\em inelastic} monopoles.)
%


For Coulomb deformations P(k) = Mn(Ek), the intrinsic reduced matrix elements
in units of $e.fm^k$  (P0 gives Mn(0) for {\em inelastic} monopoles).
These intrinsic matrix elements Mn(Ek)  use
the reduced matrix elements $\langle I ' || Ek || I \rangle$ given by the
definitions of Alder \& Winther (not by those of Brink \& Satchler),
by assuming a rotational model where $K$ is a good quantum number:
 \beqn
  Mn(Ek) = \langle I ' || Ek || I \rangle
                     / (\sqrt {2I+1}  \langle IK k0 | I ' K \rangle ) \ .
  \eeqn
Thus, for (possibly hypothetical) $0 \rightarrow k$ transitions, $ Mn(Ek)= \langle k || Ek || 0 \rangle$.\\
Mn(E2) is related to the intrinsic quadrupole moment in the rotational model by
$Mn(Ek) = Q _ 0 ~ \sqrt {5 / {16 \pi}}$.
For a uniformly charged sphere of radius $R$, to first order in $\beta_k$ we have
\beqn
  Mn(Ek) = {{3 ~ Z ~ \beta _ k ~ R^k} \over {4 \pi}}  \ .
\eeqn
The Mn(Ek) are thus of the same sign as $\beta$ and $Q_{0}$ for all rotational
bands.
\\
\bigskip
TYPES 12 \& 13, for detailed and model-independent specifications,
directly use the reduced matrix element
\beqn
  M(Ek) & = &i^{ I-I ' + |I-I ' | } ~ \langle I ' || Ek || I \rangle \\
     & =& \pm \sqrt{ (2I+1) ~ B(Ek, I \rightarrow I ' ) } \ ,
\eeqn
where the sign convention has been chosen to make M(Ek) real and symmetric under $I \leftrightarrow I'$ interchange.
The diagonal reduced matrix element gives the spectroscopic quadrupole moment
\beqn
Q_2 = \sqrt {{16 \pi} /5} ~ (2I+1)^{-1/2} ~ \langle II 20 | II \rangle ~\langle I || E2 || I \rangle \ .
\eeqn
The intrinsic and reduced matrix elements are therefore connected by
\beqn
 M(Ek) = Mn(Ek) ~ (-1)^{ [I-I ' +|I-I ' |]/2} ~ \sqrt{2I+1}  \langle IK k0 | I ' K \rangle
\eeqn
  in the rotational model where $K$ is a good quantum number,
 for states of spin $I,I'$, projection $K$, charge $Z$, and radius $R$
(DEF(k) = $\beta_k \times R$ for such states).

So Mn(Ek) is the square root of $B(Ek, 0 \rightarrow k)$ with some sign
within $K=0$ bands. Mn(Ek) always has the same sign as $\beta$.

{\em
In Other Cases, Do {\bf Not} Assume That All Transitions Have The Same M(Ek) Values.
Thus Mn(Ek) Are The Same For All Pure Rotor Transitions, In Contrast To
M(Ek).}
 %
In order to guide the user here,
effective $\beta$ values are calculated from the given Mn(Ek) or M(Ek), for
each possible transition.
{\em From version FRV onward, Mn(Ek) is used in place of M(Ek) for TYPE
10 \& 11 potentials.}

\subsubsection{Pairwise couplings}
For TYPEs 12--17, namelists {\bf \&step} are read to determine the details of the
required couplings between each pair of excited states
and of the reorientation terms for each individual state.
\begin{description}
\item[\&step] \bf IB,I A, k, STR /
\end{description}
    to put in a coupling {\em to} state IB {\em from} state IA,
of multipolarity k and strength STR.
These are terminated by an empty {\bf \&step} namelist, or by IB$<$0 indicating the last entry.\\

For each k used here, P(k) must have been non-zero,
and the required coupling has the same form factor but with relative strength
STR / P(k).   That is, STR is the reduced matrix element $<IB | M(Ek) | IA>$
= M(Ek, IA$\rightarrow$IB)
for Coulomb transitions ({\bf NOT} Mn(Ek)),
and is the `{\em reduced} deformation length'
RDEF(k) for nuclear transitions.

This `reduced deformation length' in the rotational model is
\beqn
   RDEF(k, I \rightarrow I')
         &=& DEF(k) (-1)^{ [I-I'+|I-I'|]/2}  \sqrt{2I+1} ~ \langle IK k0 | I'K\rangle\\
         &=& M(Ek) * 4\pi / [ 3 Z R^{k-1} ]
\eeqn
but may in general be varied independently of DEF(k) and M(Ek).
Both these nuclear and Coulomb interactions are thus designed to give
channel couplings which are {\em in}dependent of
the projection K values read in the \&state namelists.
This of course requires making RDEF(k) and M(Ek) dependent on the
spins of the excited states, not on any simple model.
Note: monopole potentials are included automatically, even if not listed in a  {\bf \&step} namelist.

\bigskip
The SHAPEs for deformed potentials (TYPE=10 to 13) are constructed by:

SHAPE
%\\
     = 0 to 6 : not used 
\\
     = 7,8,9 : Read from external file, as above for TYPE=1.
The multipoles are then multiplied by P(k) (by 1.0 for k=0), and
should end up as scaled to replace the form factors calculated
below:
\\
= 10 : Coulomb multipoles have the usual simple form for a
deformed charged sphere to produce M(Ek) = P(k),
varying at large distances as
\beqn
      F(r) = M(Ek) ~ e^2 ~ {\sqrt{4\pi}   \over  (2k+1)}~  r^{-k-1}
\eeqn
where $e^2=1.440$, and where this form factor will later be multiplied
by the charge number of the opposing nucleus.
\\
= 10 : Nuclear multipoles are the deformation length DEF(k) = P(k)
times the derivative of the undeformed potential times $-1/\sqrt{4\pi}$:
\beqn
      F(r) = - DEF(k) ~  \frac{1}{\sqrt{4\pi}} ~ \frac{dU(r)}{dr}
\eeqn
The shape is the same for all multipoles k $>$ 0.
\\
= 11: Nuclear multipoles calculated by numerically deforming the radii
of the previous potential, and projecting by Gaussian quadrature
onto the required multipoles (those k $>$ 0 for which P(k) is non-zero).
For small deformation lengths, these SHAPEs are the same as for
SHAPE = 10.
\\
= 12: Same as SHAPE = 11, except that the monopole k = 0 part of
the potential is also recalculated by quadrature. A first order
correction is made to the radius to improve volume conservation.
\\
= 13: Same as SHAPE = 12, except that {\bf no} first order
correction is made to the radius to improve volume conservation.

\bigskip

For TYPEs 14--17,  {\bf \&step} namelists are again read to determine the details of the
required couplings between each pair of excited states
and of the reorientation terms for each individual state.
Only nuclear couplings are yet implemented.
Specify values { \bf IB,I A, k, STR} 
to put in a coupling {\em to} state IB {\em from} state IA,
of multipolarity $|$k$|$ and strength STR. For TYPEs=14--17, the IB and IA
can reference any combination of projectile and target states,
but k$>$0 for projectile and k$<$0 for target states.
STR refers to the RDEF(k) measure above;
the factor  $(-1)^{ [I-I'+|I-I'|]/2}$ must be included.

For TYPE=17, couplings, to all orders in RDEF, are calculated between all
pairs of projectile and target states using the matrix eigensolution
method of Kermode and Rowley, Phys. Rev. {\bf C48} (1993) 2326.
The SHAPE and P0-P6 parameters on the TYPE=17 \&pot namelist are not used.

\newpage
\subsection{\&OVERLAP namelists for one- and two-particle form factors}
%


Read in a namelist for each desired particle-nucleus form factor, then an empty namelist.

\begin{description}
\item[KN1, KN2, IC1, IC2, IN]  ~\\
 %
Each form factor is indexed by a number KN, which may refer to
a single form, or for two-nucleon form factors will refer to a range
of forms KN1 - KN2
(one for each distance and angular momentum between the two nucleons).

Each form factor is the binding of one or two particles to a
specified nucleus, where the composite system is another specified
nucleus.

To specify the core and composite nuclei, their partition numbers
IC1 and IC2 are required (either order), with $|$IN$|$=1 for projectile
and $|$IN$|$=2 for target nuclei.

The mass of the particle is the strict difference of the masses of
the core and composite nuclei, except that if IN $<$ 0, then a
relativistic correction is made for effect of the relative Q-values
on the mass of the composite nucleus, when extracting by differences
the mass of the bound fragment.


\item[KIND]
the kind of coupling order (0 to 4 for one-particle states, and
6 to 9 for two-particle states) :

= 0 for (LN,SN) JN couplings   ~ ~ ~$\leftarrow$ use for typical transfers\\
= 1 for $|L_{n}, (SN,J_{core})S ; J_{com}\rangle$  (IA \& IB must be given) *\\
= 2 for eigenstate in deformed potential (fixed SN, K, Parity)  *\\
= 3 for $|(LN,SN)Jn, J_{core}; J_{com}\rangle$    (IA \& IB must be given)\\
= 4 for form of leg of the Dalitz-Thacker Triton\\
= 5  (not used)\\
= 6 for $|L_{nn}, (\ell,S_{12})j_{12}; J_{12}\rangle$  with isospin $(.5,.5) T$\\
= 7 for $|(L_{nn},\ell)L_t, (S_{12},J_{core})S_t ;  J_{com}\rangle$ \& $|(.5,.5)T,T_{core};T_{com}\rangle$  *\\
= 8  (not used)\\
= 9 for $(L_{nn},(\ell,S_{12})j_{12})J_{12}, J_{core}; J_{com}\rangle$ \&
$(.5,.5)T,T_{core};T_{com}$

where
\\  $J_{core}$ = spin of core nucleus (state Ia if given)
\\  $J_{com}$  = spin of composite nucleus (state IB if given)
\\  $K_{core}$ = projection K of core nucleus (state IA)
\\  $K_{com}$  = projection K of composite nucleus (state IB)
\\  $T_{core}$ = isospin of core nucleus (state IA)
\\  $T_{com}$  = isospin of composite nucleus (state IB)


and  *  signifies that transfers using these KINDs are not yet implemented.

\item[CH1, NN, L, LMAX, SN, IA, JN, IB] ~\\
%
CH1 = single-character identifier to distinguish clusters of
nucleons of different structures that are not further described
but should not be confused with each other.
Use A-M for positive parity clusters, and N-Z for negative parities.

NN = number of nodes (include the origin, but not infinity, so NN $>$ 0)

L  = LN = angular momentum of bound cluster relative to the core

LMAX = maximum value of L in states in deformed potential,

SN
      = intrinsic spin of bound nucleon (one-particle states)
\\  = total angular momentum (L+S) of bound cluster (KINDs 6 \& 9)
\\  = combined cluster and core intrinsic spins     (KIND 7)

IA = index (within core partition) of excited state of core,
or zero if to be specified later.

 JN
     = vector sum LN + SN,
\\   but for KIND=1, JN = SN + J$_{core}$  (i.e. S in LS coupling)
\\    and for KIND=7, JN = $L_{nn} + \ell$     (i.e. L in LS coupling)


IB = index (within composite partition) of excited state of composite,
or zero if to be specified later.

\item[KBPOT, KRPOT, BE,   ISC, NK, ER] ~\\
%
KBPOT = index KP of potential in which to bind this state

KRPOT = index KP' of potential with which to multiply this states'
wave function for transfer interactions. If zero, use KBPOT.
If the binding potential was adjusted for a specific binding
energy, then this adjusted potential (not the original) is used
for transfers.

BE = Binding Energy (positive for bound states, negative for
continuum bins)


\bigskip
{\bf BOUND STATES:}


 ISC
      = 0 to vary the binding energy for fixed potential,
\\  $>$ 0 to vary the TYPE = ISC component of the potential KBPOT
by a scaling factor to give binding energy BE.
\\  $<$ 0 to vary the TYPE = ISC component of the potential KBPOT,
as above, but {\bf also} to permanently rescale all the varied potential
components. This affects all later bound and scattering states using
the potential KBPOT.



\bigskip
{\bf CONTINUUM BINS:}


 ISC   (default value 2)\\
      = --2: no weighting or normalisation,
\\  = 1, 2 : weight wave functions by $\exp(-i\delta(k))$,
so they are real, before integrating over bin width.
\\  = 3, 4 : weight wave functions by $T(k)$* (useful for resonances)
\\  = --1, 1, 3 : normalise wave functions to unity (by usual square norm).
{\bf (This option is not recommended, for physics reasons!)}
\\  $\geq$ 10: use additional $k$ factor in the weighting function, with
mod(ISC,10) for above choices. Recommended for low-energy  bins.

The values ISC$>$0 give real-valued bins for single-channel states,
but not, in general, for coupled-channels bins. In that case, Fresco stores
{\em all} bound and continuum states as complex functions.
KIND=3 and 4 forward and reverse couplings are both calculated
explicitly, but not KIND=7 transfer couplings. For transfers, the
reverse couplings must be put in explicitly.\\
 %
{\bf Warning:} imaginary parts of bins give imaginary parts of long-range
Coulomb couplings that are {\em ignored} between abs(RMATCH) and RASYM
in the CRCWFN calculations (the values in the \&Fresco namelist).\\
 %
In general, ISC=2 is recommended (or ISC=4 for resonances), since then
the coupled-channels bins will be nearly real. With ISC=--2 (no phase weighting)
there will be  different results because of different weighting within the
bin. Note that all channels within a bin have the same scalar
weighting factor.


\item[IPC, NFL] ~\\
IPC = print control for further information:
%
\begin{tabbing}
\=IPC ~ ~ \=Print Iterations~ ~ \=Print Final Result ~ ~\=Print W/F\\
 \>0 \>no \>no \>no\\
 \>1 \>no \>yes \>yes\\
 \>2 \>no \>yes \>no\\
 \>3 \>yes \>yes \>yes\\
 \>4 \>yes \>yes \>no\\
\end{tabbing}
For bin states, read this table with `iterations' replaced by `intermediate
phase shifts'.


 NFL
      $<$ 0    : to write wave-function u(R)/R and potential overlap
V(R)u(R)/R to file number abs(NFL).
\\  $>$ 0 : to read a previously-written wave function from file number
NFL.
\\ These wave function files contain a comment line,
then (free format) NPOINTS, RSTEP, RFIRST,
followed by NPOINTS wf points in steps of RSTEP starting at r=RFIRST,
and then NPOINTS for the vertex function (potential*wf).
The file numbers NFL should be in the range 20--33 (see section \ref{files}).

\item[NAM, AMPL]~\\
If IA and IB are both non-zero, then there is enough
information to set up the spectroscopic amplitude now,
to  $\sqrt{NAM} \times AMPL$.
See discussion for the \&cfp namelists, for further information about these amplitudes.\\
{\bf Note:} If this is a form factor with mixed core levels,
then AMPL should be specified {\em here} rather than on a \&cfp namelist
(as \&cfp namelists require an IA specification, and such a state would
have multiple IA assignments).

If NAM = --1, then use AMPL for the mass of the bound particle,
independent of the MASSes in the \&partition namelists.
(If NAM $\geq$ 0, then the default particle mass is the difference of
the MASSes of the projectiles (IN=1) or targets (IN=2) for partitions
IC1 and IC2).

If NAM $<$ --1, then use AMPL instead of ERANGE for the range ER of the
energies of the upper
and lower boundaries of the continuum bins, and use at leastgr
NK=$5\times|NAM|$ integration steps over this range.
(These ER and NK override the input values.)

\item[DM] Set particle mass specifically, rather than using mass differences.

\item[NK, ER, E] Specific values for this continuum bin.

%\item[KEEP]

%\item[PPOWER]


\item[RSMIN, RSMAX] If specified, set the overlap form factors to zero outside the interval [RSMIN,RSMAX]


\item[NLAG, PHASE, AUTOWID] Parameters for {\sc Pluto} with bound or resonant states.


\end{description}

\subsubsection{One-particle KINDs}

\paragraph{For KIND = 0:}   (LN,SN) JN couplings.

One form factor with LN,SN, \& JN as read in.

LMAX is not used.

IA \& IB are used only if NAM \& AMPL are non-zero, to specify
spectroscopic amplitudes.

\paragraph{For KIND = 1:}  $L_{n},(SN,J_{core})S; J_{com}$     i.e. LS coupling.

Coupled form factors with sum over $L_{n}$ and $S$.

SN as read in, the intrinsic spin of the bound particle.

IA giving $J_{core}$, spin of core nucleus.

IB giving $J_{com}$,  spin of composite nucleus.

LMAX is maximum $L_{n}$ in summation.

JN   is maximum S  in summation.

NN \& L restrict the number of radial nodes of one component
wave function in the coupled set :

NN gives the number of radial nodes of the last partial wave of
angular momentum L$_{n}$ = L input.

\paragraph{For KIND = 2:}  eigenstates in a deformed potential, for fixed
parity and K projection :

Coupled form factors  $(l_{n},SN) JN,K$ , with sum over $L_{n}$ \& Jn.

SN, LMAX, NN \& L are as for KIND = 1 :

 SN as read in, the intrinsic spin of the bound particle.
 
 LMAX is maximum L$_{n}$ in summation.
 
 NN \& L restrict the number of radial nodes of one component
wave function in the coupled set.

IA is a core state with correct K$_{core}$ projection quantum number.

IB is a composite state with correct K$_{com}$ projection number, 
    so $K = K_{com} - K_{core}$

JN is the maximum Jn in the summation.

\paragraph{For KIND = 3:}   sum over coupled core and (ls)j particle states :
\\
Coupled form factors (L$_n$,SN)Jn,  J$_{core}$; J$_{com}$
        (summing over L$_n$,   Jn \& J$_{core}$).

NN is the required number of radial nodes for the component
wave function with core state IA and partial wave L$_n$ = L.

LMAX is maximum L$_n$ in summation.

SN as read in, the intrinsic spin of the bound particle.

All core states are included that can be coupled to form J$_{com}$,
using a deformed binding potential.
%s
{\em Note that such a deformed potential must be TYPE = 11,
whether the projectile or target is deformed: not 10 or 12 or 13}.

JN is the maximum Jn in the summation (single particle $l+s$)

IB gives $J_{com}$ (fixed) : spin of composite nucleus.

BE is the single-particle binding energy for core state IA.

\subsubsection{Two-particle KINDs}

\paragraph{For KIND $\geq$ 6:}  two-particle bound states are constructed out of sums
of pairs of previously-defined one-particle states,
and input parameters NN through to BE in the \&overlap namelist are given new meanings:

NN    $\rightarrow$ NPAIRS, the number of pair-products to be summed

L     $\rightarrow$ $\ell_{min}$ , minimum orbital angular momentum $\ell$.

LMAX  $\rightarrow$ $\ell_{max}$  , maximum orbital angular momentum $\ell$.


SN    $\rightarrow$ S$_{min}$  , minimum sum S$_{12}$ of the two nucleons' intrinsic spins
                 (S$_{max}$ = 1.0 always).

IA, IB   give core and composite states, as before.

JN    $\rightarrow$ J$_{12}$   , total angular momentum of the two-particle state
outside J$_{core}$.

KBPOT $\rightarrow$ T     , total Isospin  of the 2-nucleon state (0 or 1)s
(used to enforce  $\ell$ + S$_{12}$ + T = odd)

KRPOT $\rightarrow$ KNZR, the KN index to a single-particle state $u_{12}(r)$
of KIND 0 or 1, giving the N-N relative motion in the
other participating nucleus (usually in the light ion).

If KNZR $>$ 0, then just the overlap $ u(R)  =  <u_{12}(r) | U(r,R) > $ is
produced, suitable for zero-range two-nucleon transfer calculations.

BE    $\rightarrow$ EPS   , the threshold percentage to define components with
square norms sufficiently small to be omitted in the final two-nucleon
state.

      If ISC $\le$ 0, use Gaussian quadrature grid, in blocks of 6th-order
      positions, for the N-N distance RMIN to RNN ( \&Fresco namelist).
\\  If ISC $\geq$ 1, use uniform grid for the N-N distances.  Not so accurate.
\\  If ISC $\neq$ 0, print out numerical values of resulting two-nucleon wavefunction
U(r,R).

IPC controls the details printed (along with ISC as just above):\\
      $\geq$ 0 : one-line summary of U(r,R) form factor for each r.
\\  $\geq$ 1 : overall norm and rms radius of total NN state.
\\  $\geq$ 2 : contour plot of the L,$\ell$ components included.
\\  $\geq$ 3 : contour plot of the $U(r,R) \times$ interaction potential



NFL
      $<$ 0    : to write two-nucleon wave-function U(r,R) to file
number abs(NFL).
\\  $>$ 0 : to read a previously-written wave function from file number
NFL.  The values of NPAIRS, $l_{min}$, $l_{max}$, $S_{min}$ of
{\em the present}
run are ignored.  Thus NPAIRS can be set to zero.

\subsubsection{Namelist \&TWONT}
This namelist gives the details of the single-particle wave functions and their amplitudes in the sum of pairs.
%
\begin{description}
\item[NT(1:4, :), COEF(:)] ~\\
The sum over I of  COEF(I) $| (l_1,s_1)j_1, (l_2,s_2)j_2 ; J_{12},T \rangle$,
\\
 \mbox{~~~} for  $(l_1,s_1)j_1$  given by state $u_1(r_1)$ of KN1 = NT(1,I) $>$ 0,
\\
 \mbox{~~~}    and  $(l_2,s_2)j_2$  given by state $u_2(r_2)$ of KN2 = NT(2,I) $>$ 0,
\\
is then transformed into the required KIND = 6 format.

The $(r_1,r_2)$ coordinates become $(r,R)$ coordinates,
where
$r$ = distance between the two nucleons (angular spin $\ell$), and
$R$ = distance from the core to their centre of mass
     (corresponding angular momentum is L$_{nn}$.

If NT(3,I) $>$ 0, then the wave functions $u_1(r_1)\times u_2(r_2)$ are further
multiplied by $u_3(r)$ (of KN3 = NT(3,I)) before coordinate transformations.
Only the radial shape of KN3 is used, not any angular momentum numbers.

If NT(1,I) = --1, then an external form factor is read in from Fortran file
number NT(2,I), and processed using the subroutine EXTERN1.
At present, this routine is written to read triton wave functions
from the Grenoble Faddeev calculations,
only reading wave functions, {\em not} the potential $\times$
wavefunction (so for e.g. stripping, only use prior interactions).

If NT(1,I) = --2, then an external form factor is read in from Fortran file
number NT(2,I), and processed using the subroutine EXTERN2.
At present, this routine is written to read 3-body wave functions
from HH calculations, reading wave functions, {\em as well as} the
potential $\times$ wavefunction. The 'vrr' file format is assumed,
and the breakup (third) vertex function  is ignored.

If NT(3,I) $<$ 0, then the I'th component of the pair summation is simply
the product of cluster wave functions $u_1(r)  u_2(R) $
where $u_{2}$ may be KIND = 0 or 1.

NT(4,I) is not used in this version of FRESCOX.
\end{description}


\paragraph{For KIND = 6:} construct components of the form
$
              L_{nn}, ((\ell, (s_1,s_2)S_{12})j_{12}; J_{12},T\rangle
$
for all different $L_{nn}$, $\ell$, S$_{12}$ \& j$_{12}$ values permitted within the
limits set by $\ell_{min}$, $\ell_{max}$, S$_{min}$, and S$_{max}$
(J$_{12}$ and T are fixed)
by summing over pairs of single-particle wave functions.

\noindent
For all KINDs, the printout also lists:

DZ = derived charge of the bound particle (always positive).

DM = derived mass of the bound particle

K  = the wave number of the bound state asymptotically

NORM=overall square norm of this bound state.
The wave functions of the single-particle bound states are always
normalised to unity.

RMS= root-mean-square radius of this bound state

$D_{0}$ = zero-range stripping strength for transfers from this stare

$D$  = asymptotic stripping strength, as used e.g. in sub-Coulomb
transfers

\subsection{\&COUPLING namelists}
%
Now read in namelists for each kind of coupling between partitions
and/or between excited states, ending with a namelist which is empty or has ICTO=0.
Note that if deformed potentials were given as channel optical
potentials then there will already be some couplings between
excited states.

\begin{description}
\item[ICTO, ICFROM] ,KIND,  IP1,IP2,IP3,  P1,P2,  , IP4,IP5

The coupling is from all the states in partition ICFROM
to all the states in partition ICTO.

Couplings in the reverse direction are also included unless
ICTO $<$ 0, except for KINDs 1 \& 2 where finer control is allowed.

\item[JMAX, RMX]
Couplings are only active for J $<$ JMAX and Radius $<$ RMX,
(if JMAX=0, use JTMAX, and if RMX=0, use  RMATCH).

\item[KIND]
The couplings can be of 10 KINDs:

            = 1 : general spin transfer for projectile/transfer couplings\\
            = 2 : electromagnetic one-photon couplings\\
            = 3 : single-particle excitation of the projectile\\
            = 4 : single-particle excitation of the target\\
            = 5 : zero-range or LEA transfer with strength P1 and finite range radius P2 (in fm.)\\
            = 6 : LEA transfer using $D_0$. \& $D$ from bound states\\
            = 7 : finite-range transfer\\
            = 8 : non-orthogonality correction to a KIND 5,6,7 transfer\\
            = 9 : general partial-wave couplings\\
            = 10 : (spare)\\
            = 11 : Projectile-valence non-orthogonality\\
            = 12 : Target-valence non-orthogonality (not implemented)\\

After \&coupling namelists for KINDs 2,3,4,5,6,7 (\& 8 if IP2$>$0),
spectroscopic amplitudes are read in by means of {\bf \&cfp} namelists.
These coupling types are those for which ICTO \& ICFROM are
different, one being a core partition and the other a composite
nucleus. One table is used for all the amplitudes: it is indexed
(besides the partition numbers) by IN,IB,IA \& KN (see glossary),
and stores a real number A.

The table is filled cumulatively,
starting perhaps with \&overlap namelists, 
so amplitudes need only be
entered after the namelist for the first KIND of coupling in which
they appear.
(So if all the amplitudes are already entered for a certain KIND
of coupling, then the following namelist will be blank, to indicate
no more \&cfp namelists are to be read).

\paragraph{For KIND = 1:} 
\underline{General Spin Transfer}
(with local or non-local external form factors on FILE `INFILE' [default 4])
See section \ref{spintransfers} for more details.

IP1\\
~~~  = 0 : local form factor
\\   1 : non-local (two-dimensional) form factor



IP2\\
~~~  = 0 : read in real values      (only if IP3 $\geq$ 0)
\\   1 : read in imaginary values    (only if IP3 $\geq$ 0)
\\   2 : read in complex values        (only if IP3 $\geq$ 0)



 IP3\\
~~~  = 3 : read data from JLM folding program jlmP, for projectile couplings only
\\  2 : read data from JLM folding program jlmP, for target couplings only
\\   1 : read data from charge exchange program CHEX2, with appropriate scaling.
\\   0 : no jlmP or CHEX2 scaling factors, only P1, P2 and FSCALE
\\  --1 : write out typical non-local grid $(R,R')$ coordinates
\\  --2 : calculate non-local grid $(R,R')$ coordinates,
and call subroutine FFNL to calculate form factor.
\\  --3 : calculate non-local grid $(R,R')$ coordinates,
call subroutine FNLSET to calculate an initial form factor, and the call
FNLCC for each pair of coupled partial waves, for L-dependent factors.
\\ --4 : call subroutine FNLREAD to read in non-local form factor, and multiplying it by the Racah-algebra factor
in section \ref{spintransfers} (except for the $1/\sqrt{4\pi}$ factor) for given {\tt IB,IA} final and initial excited states. 
The order of form factors in the file must agree with the order called by a double loop of channels, with the `to' channel number varying more rapidly.
\\ --5 : call subroutine FNLREAD to read in non-local form factor for given {\tt IB,IA} final and initial excited states. 
The order of form factors in the file must agree with the order called by a double loop of channels, with the `to' channel number varying more rapidly.



 P1,P2 : scaling for the real and imaginary parts (respectively).\\

   If IP3 $\geq$  0, read FROM FILE `INFILE'  the following lines
   (until a blank line):
\begin{verbatim}
     Format 16.2: I4, 3F8.4,          I4,  2F4.0,   2I4,   A35
                NP ,HNP,RFS,FSCALE, LTR, PTR,TTR, IB,IA, COMMENT
            for NP = number of radial points
                HNP = step size
                RFS = radius of first point
                FSCALE = scaling factor to be applied
                LTR = L-transfer
                PTR = projectile spin transfer
                TTR = target spin transfer
                IB  = excited-state pair fed by coupling
                IA  = excited-state pair feeding the coupling
                COMMENT = comment for display in printout.
\end{verbatim}
    ~~~~~~ then free format, repeatedly until sufficient data is read.\\


 If IP1 =
      0, read local form factor from i=1 to N
\\  1, read successively for j=1,NLO the non-local form factors
 FNL(i,j), i=1 to NP   [NLO = RNL / max(HNL,HCM)]


When IP3=1 for CHEX2 input, the RFS and its F8.4 is omitted from Card 16.2,
with default value RFS=HNP.


\paragraph{For KIND = 2:}
\underline{Electromagnetic one-photon couplings} (for $E\lambda$ and
$M\lambda$ processes).



 ICTO is the gamma partition and ICFROM the particle partition.
 The photon must be on the `projectile' side, and the bound state between
 target states.

 IP1 = $\lambda$ : The multipolarity of the radiation.
 If $\lambda >$ 0, include all multipoles 1,...,$\lambda$
permitted by parity, whereas if IP1 $<$ 0, include only the multipole
$\lambda$ = abs(IP1).
%

IP2\\
~ ~   = 0,4 : Calculates both electric and magnetic couplings
(reads $g$-factors from P1\&P2)
\\    1,5 : for electric only
\\    2 : for magnetic only (reads $g$-factors from P1\&P2)
\\    4,5 : include also Siegert remnant for electric transitions


 IP3:  {\em not implemented yet}
%~ ~   = 0 : no intrinsic convection couplings
%\\    1 : intrinsic convection coupling for projectile only
%\\    2 : intrinsic convection coupling for target only
%\\    3 : intrinsic convection coupling for projectile and target
%\\    4 : turn on magnetization term for electric component


 P1  = projectile $g$-factor  \\
 P2  = target $g$-factor  \\

IP4 \\
~  ~~   = 0 : direct capture only
\\    1 : semi-direct capture only
\\    2 : direct + semi-direct capture mechanisms.


\paragraph{For KIND = 3 or 4:} 
\underline{Single-particle excitations of the projectile (3) or target (4)}


 For these KINDs, ICTO is the partition of the nuclei being excited,
and ICFROM is used to indicate the
{\em core} partition if the single particle were removed.  No couplings are
generated to or from the ICFROM partition, only within ICTO partition.

 IP1 = Q : The multipole order of the deforming potential due to
the colliding nucleus. If Q $>$ 0, include all multipoles 0,1,...,Q
permitted by parity, whereas if IP1 $<$ 0, include only the multipole
Q = abs(IP1).
%

IP2
\\   = 0 : Coulomb \& nuclear (complex)
\\  = 1 : nuclear (complex) only
\\  = 2 : Coulomb only



IP3
\\  = 0 or 10 : include all re-orientation terms
\\  = 1 or 11 : no re-orientation terms for $Q > 0$
\\  = 2 or 12 : ONLY re-orientation terms
\\  = 3 or 13 : include only couplings to and from the ground state,
    but NOT gs reorientation,
\\  = 4 or 14 : include diagonal couplings,
                and  couplings to and from the ground state
\\  = 5 or 15 : include diagonal couplings,
                and  couplings to and from any bound state
\\  $\geq$ 10      : read namelist {\bf \&scale} for
complex factors [QSCALE(Q), Q=max(0,-IP1):abs(IP1)]
to scale the folded form factors for multipoles Q.

IP4 = $Q_{\rm max}$, the max deformed core potential multipole\\
IP5 = $\Lambda_{\rm max}$, the new multipole order for formfactor reduction.\\
For these core-excitation (XCDCC) options, the code needs to be compiled with the -Dcorex option.

 P1  = FLOAT( potential KP index for fragment - target interaction) \\
 P2  = FLOAT( potential KP index for core     - target interaction) \\
        (only SCALAR parts of the potentials P1 and P2 are used).

\paragraph{For KIND = 5 or 6:}
\underline{Zero-range \& LEA transfers}

IP1 and IP2 not used.

For KIND=5,
\\  P1 = $D_0$    - ZR coupling constant
\\  P2 = FNRNG - Effective finite-range parameter for use
in LEA (in units of fm.)

For KIND=6, use $D_0$ and FNRNG = $\sqrt{(D/D_0 - 1)/k^2}$. from the projectile
bound states.  With unbound states, or if IP3 = 1,
use $D_0$ from state, but FNRNG = P2
from this namelist.

 In both KINDs, read in \&cfp namelists, noting that for KIND = 5
spectroscopic factors for the projectile are not needed, and are ignored.

Users of these interactions kinds should also pay attention to the
the parameter INH in the \&Fresco namelist.

\paragraph{For KIND = 7:}
\underline{Finite-range transfers.}


IP1 =\\
~~~  0,-2 : POST interaction
\\  1,-1 : PRIOR interaction
 (N.B. meaning of IP1 thus depends on ICTO \& ICFROM!!!)
\\  $\le$ -1 : Use $\theta$ quadrature from $\theta = \pi$ down to $\theta = 0$.
 (Useful for finite-range knock-on with light projectiles)
\\  $\le$ -3 : VCORE interaction:
Use ONLY the core-core interaction potential
 (Useful for finite-range knock-on with light projectiles)
\\ 2 : Surface transfer operator, on surface if {\em final} bound state $r' =$ P1. IP2 and IP3 ignored.
\\ 3 : Surface transfer operator, on surface if {\em initial} bound state $r =$ P1. (Not yet implemented)
\\ 4 : Surface transfer operator calculated as PRIOR--POST. Need also to set RSMAX for {\em final} bound state to the desired surface radius. 
   (Implemented, but not yet correctly).
\\ 5 : Surface transfer operator calculated as POST--PRIOR. Need also to set RSMAX for {\em initial} bound state to the desired surface radius. 
   (Not yet implemented).


IP2 =\\
~~~  0 : no remnant
\\  1 : full real remnant
\\  -1 : full complex remnant
\\  2 : ``non-orthogonality remnant" - this works by inserting
a KIND = 8 coupling namelist after this coupling, before any subsequent
couplings.
This is does not affect the one-step amplitudes, and is only useful
if another transfer step follows this coupling.



 IP3 = KPCORE : the number KP of the potential to use between
the two cores, in the remnant part of the interaction potential.

       If IP3=0, use as KPCORE the optical potential given for
the first pair of excited states in the partition of projectile core.
(this uses the observation that optical potentials tend to depend
more on the projectile than the target, and must clearly be re-examined
if the projectile is heavier than the target.)

 P1,P2 : not used.


\paragraph{For KIND = 8:}
\underline{Non-orthogonality supplement}
appropriate to a previous
KIND 5,6 or 7 interaction.


 IP1 = \\
~~~  0 : post
\\  1 : prior
 (N.B. IP1 should be the same as the previous interaction!)



IP2
\\  $>$ 0 : read in spectroscopic factors as \&cfp namelists
 (Only useful if you did NOT have a previous KIND 5, 6 or s7
interaction, which would have needed the amplitudes then.)
\\  = 0 : no \&cfp namelists to read.


Note that you should {\em either} use KIND=7, $|$IP2$|$=2,
{\em or} use  KIND=7, $|$IP2$|$=1  and a KIND=8.
If you have KIND=7, $|$IP2$|$=2 and a KIND=8 namelist, then this is
double counting.
\bigskip


\paragraph{For KIND = 9:} 
\underline{General Partial-Wave Couplings}
(local or non-local external interactions from FILE `INFILE' [default 4])

IP1 \\
~~~  = 0 : local form factor
\\   1 : non-local (two-dimensional) form factor

IP2\\
~~~  = 0 : read in real values   
\\   1 : read in imaginary values   
\\   2 : read in complex values       

 P1,P2 : scaling for the real and imaginary parts (respectively).\\

For IP1=0 (local), read from file `INFILE' the following lines
   (until a blank line):
\begin{verbatim}
    Line (free format)           NP, sHNP, RFS
            for NP = number of radial points
                HNP = step size
                RFS = radius of first point
\end{verbatim}
For IP1=1 (nonlocal), read from file `INFILE' the following lines    (until a blank line):
\begin{verbatim}
    Line (free format)           NLLI, RINTPI, NLOI, HNLI, NLCI
            for NLLI = number of radial points in R
                RINTPI = step size in R
                NLOI = number of radial points in D=R'-R
                HNLI= step size in D
                NLCI=offset index for D=0
\end{verbatim}

\begin{verbatim}
   Lines:
F8.1,I4,       2I4,f6.1,      2I4,F6.1       L2
JTOTAL,PARITY, IB,LVAL1,JVAL1,IA,LVAL2,JVAL2,REV
\end{verbatim}
each with following potential data of Cards 16.6 or 16.7.
These are the couplings from partial wave {\em to}  (LVAL1,JVAL1) for excited state pair IB
{\em from} partial wave (LVAL2,JVAL2) for excited state pair IA, in coupled channels set JTOTAL of parity PARITY (+1 or --1). 
The reverse coupling is also included if REV is true.

If IB or IA is zero, match any state. If JVALi $< $0, then match any JVALi. If PARITY=0, match any parity.
If JTOTAL$<$0, match any JTOTAL.  If LVALi $<$ 0, then match any LVALi.
If LVAL1=LVAL2$<$0, then require partial wave coupling to be diagonal for equal LVAL=--LVALi-1.


For IP1=0 (local), \\
  \hspace*{8mm} {\tt read (free format) NP values}

For IP1=1 (nonlocal), read NLLI loops of:\\
      \hspace*{8mm} {\tt read NLOI values (free format)}.

Go back to reading the above input files (though subsequent values in those files are ignored).


\paragraph{For KIND = 11:}
\underline{Projectile-valence non-orthogonality}
Construct non-orthogonality overlap, for use in R-matrix or Lagrange-mesh methods,  for the case when a 
projectile $p$ scattering on a target consisting of \{core $c$ + valence $v$\} can be rearranged in another partition 
to be projectile$'$ $v$ scattering on a target$'$ consisting of \{core $c$ + valence $p$\}.

\noindent
ICTO, ICFROM are the two partitions with $p$ and $v$ as projectiles (either order).

\noindent
IP1 = ICORE is the partition in which the target is the bare  core  $c$.

Spectroscopic amplitudes need to have been defined for $\langle c | c+v = t \rangle$  and $\langle c | c+p = t'  \rangle$. 
%Any amplitudes defined after this KIND=11 \&Coupling are for $\langle$ ICORE $|$ ICTO $\rangle$ overlaps. \\
NO amplitudes in \&cfp namelists are to be read in after this coupling.

The reverse case (KIND=12), where the projectile is composite rather than the target, is not yet implemented.
\end{description}

\subsubsection{\&CFP namelists}
  {\em Spectroscopic amplitudes} {\rm for the overlaps between
partitions ICTO \& ICFROM already defined by a \&coupling namelist}:
\begin{description}
\item[IN, IB, IA, KN,  A] ~\\
%
meaning that the overlap of the composite nucleus in excitation
state IB with the core nucleus in excitation state IA is the
bound-state form factor KN with amplitude A. The IN=1 specifies
projectile overlap, and IN=2 target overlap.

 If the form factor KN mixes different IA levels, then the spectroscopic
 amplitudes should rather be specified in an \&overlap namelist.

N.B. The amplitudes A are signed, and are NOT the spectroscopic
factors, but will typically be the square roots of these factors.
For transfers out of or into closed shells of N antisymmetrised nucleons,
the spectroscopic factors will usually contain factors of N,
so the spectroscopic amplitudes needed by FRESCOX will typically 
need to already contain factors of $\sqrt{N}$.
\\
The sign of A should be consistent with the spin coupling order used in
the program, which is
\beqn
  (\ell,s)j, J_{core}; J_{com}
\eeqn
for binding a $\ell sj$ nucleon onto a core of spin $J_{core}$ to form $J_{com}$.
\\
 If IN$<$0, use abs(IN) in this the last \&cfp namelist\\
 If IN=0, no more \&cfp namelists to be read.
\end{description}

%\subsection{Additional data in Fortran input file 4:}
%%
%All these data involve reading real or complex floating point numbers
%by repeating
%\begin{verbatim}
%  FORMAT(6E12.4)
%\end{verbatim}
%
%Line A1 :  External Potential Form Factors (SHAPEs 7-9 on Cards 10)
%
%Lines A2 :  External Coupling Potentials for KIND 1 \& 2 couplings

\subsection*{Glossary of Indexing Variables}
%

IC,IC1,IC2    index to mass partitions

IA,IB         index to projectile-target pairs of excited states
              within each partition

KP            index to table of potentials, for scattering, binding,
              and folding

KN            index to single-particle bound states

KN1,KN2       two-particle bound states are given for a range (RMIN to
              RNN) of internucleon separations, and are stored in the
              MSP table from KN1 to KN2 inclusive.

CP            index to the coupling set up

\newpage
\section{Input CDCC format}

{\bf \&CDCC namelist}

{\tt
hcm, rmatch, rintp, hnl, rnl, centre,  rsp, iter, pset, llmax,  dry,
rasym, accrcy, switch, ajswtch, sinjmax,  cutl, cutr, cutc,
absend, jtmin, jump, jbord, nnu,  rmatr, nrbases, nrbmin,  pralpha, pcon,  meigs,
listcc, smats, veff, chans, xstabl, thmin, thmax, thinc,
smallchan, smallcoup, melfil, nosol, cdetr, numnode, treneg,
nlpl, trans, pel, exl, cdccc, qscale, pade, kfus,
elab,  lab, lin, lex, \\
 hktarg, ncoul, reor, pauli,  nk,
q, ipc, iscgs, ipcgs,  hat, remnant, postprior, quasi,
sumform, qc, la, static, expand, maxcoup
}\\

\vspace{-3mm}

are all the same as the {\bf \&FRESCOX} namelist, except for:\\
{\tt cdccc} which is an alias for {\tt cdcc} of \&Fresco namelist,\\
 %
{\tt q} = {\tt ip1}, multiple for the projectile single-particle couplings,\\
{\tt ncoul} = {\tt ip2}, selecting nuclear and/or coulomb,\\
{\tt reor} = {\tt ip3}, selecting diagonal and/or off-diagonal couplings,\\
{\tt qc} = {\tt ip4}, $Q_{\rm max}$ for the deformed core potential multipoles,\\
{\tt la} = {\tt ip5}, $\Lambda_{\rm max}$ for the new multipole orders with formfactor reduction,\\
{\tt hat} (logical, default T) to use mean bin energies (otherwise midpoint),\\
{\tt quasi}: if assigned, set all channel energies as if for this projectile
energy (eg -{\tt be} for adiabatic),\\
{\tt iscgs,ipcgs} = {\tt isc,ipc} (\&overlap) for the projectile ground state wave
function,\\
{\tt elab} = {\tt ELAB(1)}, just the one projectile lab energy,\\
{\tt hktarg} = target value of $h.K$, where $h$={\tt HCM} and $K$ is the elastic wave number.
(default {\tt hktarg}=0.2). \\
\hspace*{1cm} If {\tt HCM}=0 in \&Fresco namelist,
then $h$ is calculated from {\tt elab} and {\tt hktarg}.\\



{\tt sumform} determines the bin formfactor reductions
(default 2 if there are any {\tt kind}=3 bin states, else 0):\\
~ ~   = 0 : no formfactor reduction:
     $\langle {\tt KN} | K Q \lambda | {\tt KNP} \rangle$
\\    = 1 : $KQ\lambda$ formfactors summed into new multipole $\Lambda$:
     $\langle {\tt KN} | \Lambda | {\tt KNP} \rangle$
\\    = 2 : formfactors summed into composite projectile state:
     $\langle {\tt IB} | {\Lambda} | {\tt IBP} \rangle$.
      This is not allowed if have spin-orbit or transfer couplings, and
      {\tt sumform}=1 will be set in these cases.\\

If have no cc bins, then $K$ multipoles truncated using {\tt ip1=q}.\\
If have cc bins and $Q_{\rm max}${\tt =qc=ip4} and
$\Lambda_{\rm max}${\tt =la=ip5} are unset, then
{\tt q} sets the maximum multipole order of $\Lambda$, and
 all possible values of  $KQ\lambda$ are used that couple to $\Lambda \le {\tt q}$\\
If {\tt qc} and {\tt la} are both set, then control each multipole
individually:\\
\hspace*{1cm} {\tt q}     = $K_{\rm max}$,  {\tt ip1} set in usual way\\
\hspace*{1cm} {\tt ip4=qc}  = $Q_{\rm max}$. \\
\hspace*{1cm} {\tt ip5=la} = $\Lambda_{\rm max}$ maximum order for new
multipole.\\
\hspace*{1cm}  Note: $\lambda$ will always run from $0\rightarrow Q$.\\


{\tt trans} determines the number of transfer partitions:\\
~ ~   = 0 : no transfers: no E(jectile) or R(esidue)
\\    $<$ 0 : only R(esidue): the Ejectile and Core are identical
\\    $>$ 0 : both E(jectile) and R(esidue) independently of Core nucleus.

{\tt postprior} = {\tt ip1} for finite-range transfer couplings,\\
{\tt remnant} = {\tt ip2} for finite-range transfer couplings,\\
{\tt pauli} = attempted Pauli blocking using non-orthogonality couplings.


\noindent{\\ \bf \&NUCLEUS namelist}\\
{\tt part,name,mass,charge,spin,parity,be,n,l,j,ia,a,kind,lmax,nch,nce,ampl
}\\
 %
where this line is repeated for each {\tt part} beginning P: projectile, C:
core, V: valence, T: target, E: ejectile, R: residue. The number of nuclei
(4, 5 or 6) depends on {\tt trans}: see above.\\
 %
{\tt name,mass,charge} as {\tt name, mass, zp/t} in a \&partition namelist,\\
{\tt spin,parity} as {\tt Jp/t, Bandp/t} in a \&states namelist.\\
{\tt be,n,l,j,ia,a,kind,lmax,nch} prescribe the projectile P bound state,
and with transfers also for residue R.\\
{\tt nce} is the number of Core excited states.

If {\tt nce}$>$0, then read that number of {\bf
\&CORESTATES} namelists. \noindent{\\ \bf \&CORESTATES namelist}: {\tt
spin,parity,ex }\\ for spin, parity ($\pm1$) and excitation energy of each Core
state above the ground state.

\noindent{\\ \bf \&BIN namelist}\\
{\tt
spin,parity, step,start,end,energy, n,l,j,isc,ipc, kind,lmax,nch,ia,il,ampl
}\\
These are repeated until an empty  {\bf \&BIN} namelist is encountered
({\tt step}=0).\\
Each bin set has the same {\tt spin, parity, l, j, isc, ipc, kind, lmax, nch, ia,
il, ampl}, but a different energy.
The energy range is divided into ({\tt end-start})/{\tt step} bins.
If {\tt energy} then these are evenly spaced in energy, else they
are evenly spaced in momentum $k \propto \sqrt{E}$.
Changing or starting partial waves is equivalent to {\tt start}= 0.001.
\\
The values of {\tt lmax, nch, ia, il} are only needed coupled channels bins
{\tt kind}=3, with {\tt il} being the channel number of the incoming partial wave,
which, if {\tt il}=0, is defined as the channel with quantum numbers
{\tt l, j} and {\tt ia}. The array {\tt ampl} gives overall multiplicative
spectroscopic amplitudes to the bin. The values of {\tt isc, ipc, kind, lmax}
have the same meaning as in a \&overlap namelist, with {\tt j} an alias for {\tt jn}.\\
If {\tt l,j,ia} are not set then {\tt kind}=3 coupled-channels bins are generated
using {\tt lmax}, for all incoming waves. Setting {\tt il} will select an
incoming channel.

\noindent{\\ \bf \&POTENTIAL namelist}\\
{\tt
part, a1, a2, rc, ac, v, vr0, a, w, wr0, aw,
wd, wdr0, awd, vso, rso0, aso, shape, freal, fimag,
vsot, rsot0, asot, l, parity, nosub, itt,
beta2, beta3, idef, beta2c, beta3c, beta2m, beta3m
}\\
where this namelist is repeated for each {\tt part} beginning
P: projectile-target  optical potential, C: core-target optical potential,
V: valence-target  optical potential,
T: projectile (C+V) ground state,
B: projectile channels {\em not} containing the ground state,
T: transfer channel optical potential,
E: ejectile bound state, R: residue bound state.\\
Different {\tt part}=B(in) potentials may be defined depending on {\tt parity}
or partial wave {\tt l}. \\
{\tt nosub} means that the P(rojectile) optical potential is {\em added} to
the CDCC couplings as a diagonal in all projectile state channels.\\

For deformations, {\tt beta2m, beta3m} are the nuclear {\em fractional}
deformations, and {\tt beta2c, beta3c} are the Coulomb equivalents (both with default
values  {\tt beta2, beta3} respectively).\\
The Coulomb and nuclear deformations may also be restricted by {\tt idef}:
\\   = 0 : Coulomb \& nuclear (complex)
\\  = 1 : nuclear (complex) only
\\  = 2 : Coulomb only


{\bf Sample CDCC input file:}
\begin{verbatim}
11Be+4He spdf; 1+5*10+2*5 chs 0-10 MeV, q=0-3 2200 MeV, 30/100 fm
CDCC
 &CDCC
   hcm=0 rmatch=-30 absend=-50 rasym=100 accrcy=0.001
   elab=2200
   jbord=  0   60 200 2500
   jump =  4    5  20
   thmax=30 thinc=.05 smats=2 xstabl=1  cutr=-10 cutc=0
   nk=50 ncoul=0 reor=0 q=3
   /
 &NUCLEUS part='Proj' name='11Be' spin=0.5 parity=+1 be = 0.500 n=2 l=0 j=0.5 /
 &NUCLEUS part='Core' name='10Be' charge=4 mass=10 /
 &NUCLEUS part='Valence' name='neutron' charge=0 mass=1 spin=0.5/
 &NUCLEUS part='Target' name='4He' charge=2 mass=4 /

 &BIN spin=0.5 parity=+1 step=0.5 end=10. energy=F l=0 j=0.5/
 &BIN spin=0.5 parity=-1 step=0.5 end=10. energy=F l=1 j=0.5/
 &BIN spin=1.5 parity=-1 step=0.5 end=10. energy=F l=1 j=1.5/
 &BIN spin=1.5 parity=+1 step=1.0 end=10. energy=F l=2 j=1.5/
 &BIN spin=2.5 parity=+1 step=1.0 end=10. energy=F l=2 j=2.5/
 &BIN spin=2.5 parity=-1 step=2.0 end=10. energy=F l=3 j=2.5/
 &BIN spin=3.5 parity=-1 step=2.0 end=10. energy=F l=3 j=3.5/
 &BIN /

 &POTENTIAL part='Proj' a1=11 a2=4 rc=1.0  /
 &POTENTIAL part='Core' a1=10 a2=4 rc=1.0
            V=46.92 vr0=1.204 a=0.53 W=23.46 wr0=1.328 aw=0.53 /
 &POTENTIAL part='Valence' a1=4 rc=1.3
            V=37.14 vr0=1.17  a=0.75 W=8.12  wr0=1.26  aw=0.58 /
 &POTENTIAL part='Gs' a1=10 v=51.51 vr0=1.39 a=.52 vso=0.38 rso0=1.39 aso=0.52/
 &POTENTIAL part='Bi' a1=10 v=28.38 vr0=1.39 a=.52 vso=0.38 rso0=1.39 aso=0.52/
\end{verbatim}
\newpage
\section{Input SFRESCOX format}

This is an program which search for a $\chi^2$ minimum when comparing
the outputs of FRESCOX with sets of data, using the {\sc Minuit} search
routines.

The inputs for {\sc SFrescox} specify the FRESCOX input and output files,
the number and types of search variables, and the experimental data sets.

\subsection{Search file}
A {\tt search\_file} format is defined as follows:
\begin{verbatim}
(2*string, 2*integer: free format)
input_file, output_file, nvariables, ndatasets
\end{verbatim}
\smallskip

{\tt input\_file}: name of file (in single quotes) with regular FRESCOX input\\
{\tt output\_file}: name of file for the final FRESCOX output\\
{\tt nvariables}: number of search variables\\
{\tt ndatasets}:  number of experimental data sets.\\

Repeated {\tt nvariables} times, a namelist with some of the following variables:
\begin{verbatim}
namelist  &variable:
name,kind,step,valmin,valmax, kp,pline,col,potential,nopot,
dataset,datanorm,dataEshift,reffile, nafrac,afrac,
energy, damp, jtot,par,channel,width,rwa,B, leff,
pline2,col2,ratio2
\end{verbatim}
{\tt name} is name of variable (up to 10 characters),\\
{\tt kind}: 0=ignore, 1=potential, 2=afrac, 3=R-matrix energy,\\
   \hspace*{10mm} 4=R-mat partial width, 5=dataset normalisation\\
{\tt step}: step size for finding derivatives etc, zero for fixed. (default 0.01)\\
{\tt valmin}: strict lower bound (if non-zero),\\
{\tt valmax}: strict upper bound (if non-zero), \\
{\tt null}: (default -124578) `undefined' value,\\

{\tt kind=}{\bf 1: Variable potential parameter:}\\
{\tt kp}: number of potential as in \&pot namelist,\\
{\tt pline}: order of which \&pot namelist within potential definition,\\
{\tt col}:   number within namelist: selecting P0--P7 ,\\
{\tt potential}: value of this potential parameter (default {\tt null}: use {\tt input\_file})\\
{\tt pline2,col2}: if both assigned, then potential parameter at line {\tt pline2} and index {\tt col2} is automatically set to 
a value of {\tt ratio2} times this parameter's value. \\

{\tt kind=}{\bf 2: Variable spectroscopic amplitude:}\\
{\tt nafrac}: order number of \&cfp namelist in {\tt input\_file}\\
{\tt afrac}: spectroscopic amplitude A (default {\tt null}: use {\tt input\_file})\\

{\tt kind=}{\bf 3: Variable energy of R-matrix term:}\\
{\tt term}: term number (default 1)\\
{\tt jtot}: J value for additional R-matrix term\\
{\tt par}: parity (+1 or --1)\\
{\tt energy}: energy (cm MeV in entrance channel) of additional R-matrix term (default 0)\\
{\tt damp}: energy (cm MeV in entrance channel) of imaginary width of this R-matrix term (default 0). The actual  imaginary part is {\tt damp}/2\\
{\tt nopot}: Disable potential \& Buttle correction for this J/pi set (logical T or F, default F)\\

{\tt kind=}{\bf 4: Variable partial width of R-matrix term:}\\
{\tt channel}: channel number (in order generated by FRESCOX)\\
{\tt term}: term number (default 1)\\
{\tt rwa}: the {\tt width} is reduced width amplitude $\gamma$, else $\Gamma=2\gamma^2 P$.  (default T)\\
{\tt B}: the boundary condition {\tt B}. (Default: from {\tt BNDX}).\\
{\tt width}: width for this channel (MeV$^{1/2}$ if {\tt rwa} or energy subthreshold, else MeV(cm)) (default 0)\\


{\tt kind=}{\bf 5: Variable dataset normalisation:}\\
{\tt dataset(1:2)}: index (or, first and last, inclusive) of which dataset (1 to {\tt ndatasets}) (default 1), or\\
{\tt reffile}: filename of dataset. If has a `*', renorm all sets with names matching up to that character,\\ 
{\tt datanorm}: search on absolute normalisation factor of this dataset and its error (default: 1.0)\\


{\tt kind=}{\bf 6: Variable dataset energy shift:}\\
{\tt dataset(1:2)}: index (or, first and last, inclusive) of which dataset (1 to {\tt ndatasets}) (default 1), or\\
{\tt reffile}: filename of dataset. If has a  `*', shift all sets with names matching  up to that character,\\ 
{\tt dataEshift}: search on absolute energy incremental shift of this dataset  (default: 0.0)\\

{\tt kind=}{\bf 7: Variable damping width of R-matrix term:}\\
{\tt term}: term (default 1). For the real energy there should be a {\tt kind}=3 variable with the same {\tt term}.\\
{\tt damp}:  Imaginary width (cm MeV in entrance channel) of this R-matrix term (default 0). \\%he actual  imaginary part is {\tt damp}/2\\
When {\tt energy} is  set, use the (real)  {\tt leff} to rescale the damping width according to energy above threshold {\tt energy} 
    by the penetrability ratio $P_{\rm leff}(E - {\tt energy})/P_{\rm leff}(e_{\rm pole} - {\tt energy})$, with elastic Coulomb. 
    If not set, use constant damping widths as  2*imaginary parts of the pole energies. \\
The {\tt damp} variable in {\tt kind}=3 is fixed, but those in {\tt kind}=7 can be varied in searches.\\



Repeated {\tt ndatasets} times, namelist with some of the following variables, then the
corresponding data:
\begin{verbatim}
namelist \&data:
type,data_file,points,delta,xmin,lab,energy,angle,
idir,iscale,abserr,ic,ia,k,q,jtot,par,channel,value,error,leg,ib,
pel,exl,labe,lin,lex,term
\end{verbatim}

{\tt type} (default 0)
\\ = --3 Legendre coefficient for input energy and fixed order {\tt leg}
\\ = --2 Legendre coefficient (integer order {\tt leg}) for input energy: double distribution
\\ = --1 Legendre coefficient for varying integer order {\tt leg}, for fixed energy
\\ = 0 angular distribution for fixed energy
\\ = 1 excitation and angular cross section double distributions
\\ = 2 excitation cross section for fixed angle
\\ = 3 excitation total cross section. Also: ic=0: {\tt ia}=0 is total reaction cross section;
{\tt ia}=1 is total fusion cross section; 1$<${\tt ia}$<$NFUS+1 are fusion from potential
KFUS (in \&Fresco namelist) for inelastic state {\tt ia}--1.
\\ = 4 excitation phase shift for fixed partial wave
\\ = 5 desired factor for bound state search (binding energy or potential scale
 according to ISC).
\\ = 6 specific experimental constraint on a search parameter {\tt par},
    to be {\tt value} with error {\tt error} ({\tt abserr}=T or F).
\\ = 7 specific constraint on energy of R-matrix pole {\tt term} in the Brune basis 
    to be {\tt value} with error {\tt error} ({\tt abserr}=T or F).
\\ = 8 specific constraint on total formal width of R-matrix pole {\tt term} in the Brune basis 
    to be {\tt value} with error {\tt error} ({\tt abserr}=T or F).


{\tt data\_file}: name of data file with data, `$=$' for {\tt search\_file},
 `$<$' for stdin (default `$=$')\\
{\tt points}: number of data points (default: keep reading as many as possible),
\\
{\tt delta}: if non-zero, construct linear x-scale from {\tt xmin} in steps of {\tt delta},
 (default 0)\\
{\tt lab}: T  for lab angles and cross sections (default F for cm frame).\\
{\tt Aflip}: T to change cm calculated angles to $180-\theta$ (default F).\\
{\tt energy}: lab energy for this {\tt type}=0 dataset (default: use {\tt ELAB(1)} from \&Fresco namelist)\\{\tt pel,exl,labe,lin,lex}: specify input channels for this data set.

{\tt idir} (default 0)
\\ =-1 cross-section data are given as astrophysical S-factors, but will be converted to absolute,
\\ = 0 cross-section data are given in absolute units,
\\ = 1 cross-section data are ratio to Rutherford
\\ = 2 cross sections are given in absolute units but will be
         converted to ratio to rutherford.



{\tt iscale} (default 2)
\\ = -1 dimensionless (eg ratio to rutherford if {\tt idir}=1),
\\ = 0 absolute cross-section units are fermi-squared/sr.
\\ = 1 absolute scale is barn/sr
\\ = 2 absolute scale is mb/sr
\\ = 3 absolute scale is micro-b/sr


{\tt abserr}: T or F for absolute error (default F)\\
{\tt ic}: partition number of cross section channel (index of Cards 6)  (default 1)\\
{\tt ia}: excitation pair number of cross section (index of Cards 7)  (default 1)\\
{\tt k}: tensor rank of cross section (0, 1, 2,..) (default 0). \\
{\tt q}: index for tensor $T_{kq}$ of cross section (0, 1, 2,..) (default 0).
    $T_{00}$ = angular distribution $\sigma(\theta)$.\\
{\tt jtot}: The {\tt J} value (as on Card 2) for phase shift\\
{\tt par}: The parity (--1 or +1) for phase shift, for {\tt type}=4\\
{\tt par}~~~~~~ or number of search parameter, for {\tt type}=6.\\
{\tt channel}: Channel number of entrance channel in coupled channels set (default 1),
for phase shift\\
{\tt value}: Value of search parameter, for {\tt type}=6\\
{\tt error}: Error in value of search parameter, for {\tt type}=6\\
~\\

{\tt ib}: if 0$<${\tt ib}$<${\tt ia}, then {\tt ib} is final state of gamma decay. (default 0)\\
     If this is given for data {\tt type}=0,1,2, then the {\tt ia} is the initial state, and 
     all angles are the {\em  gamma} detection angles, and cross sections are integrated over all ejectile particle angles.
     Need to set {\tt dgam}=1, {\tt pp}=3, {\tt kqmax}$\geq 2\times$spin of state {\tt ia}, 
        {\tt thmin}=0 and {\tt thmax}=180 degrees.\\`

{\bf Data input} (free format):
\\ if {\tt type}=1 or --2, read:  {\tt energy, angle, value, error}
\\ if {\tt type}=5, read:  {\tt kn, target, error}
\\ ~~~~otherwise:
\\ if {\tt delta}$\ne$0, read:  {\tt value, error}
\\ if {\tt delta}=0, read:  {\tt x, value, error}


~~~~~ where {\tt x} is angle for {\tt type}=0, is {\tt leg} for {\tt type}=--1 or --2, and lab energy for {\tt type}=--3, 2, 3 and 4.\\

\subsection{Sfrescox commands}

%\subsection*{{\tt SFresco} Input}
The inputs to {\tt SFresco} are\\
Card 1: name of {\tt search\_file}\\

Lines 2: repeatedly one of (in all upper or all lower case):
\\ EX: exit
\\ Q: query status of search variables
\\ V: query status of unfixed search variables
\\ CHA: query status of changed search variables
\\ CHI: list $\chi^2$ status by dataset
\\ SET {\tt var val}: set variable number {\tt var} to value {\tt val}.
\\ FIX {\tt var}: fix variable number {\tt var} (set {\tt step}=0).
\\ FXWID : fix all R-matrix widths variables.
\\ FXRPE : fix all R-matrix pole energy variables.
\\ FXPBG : fix all pole energies variables with 'BG' in their name.
\\ FXBG : fix all variables with 'BG' in their name.
\\ FXNOR : fix all dataset norm variables (kind=5)
\\ FXSHF : fix all dataset energy shift variables (kind=6)
\\ FREE : free all R-matrix energies and widths variables.
\\ STEP {\tt var step}: unfix variable number {\tt var} with step {\tt step}.
\\ SCAN {\tt var val1 val2 step}: scan variable number {\tt var}
from value {\tt val1} to value {\tt val2} in steps of {\tt step}.
\\ SHOW: list all datasets with current predictions and $\chi$ values.
\\ PLOT {\tt plotfile}: write file (default: {\tt search.plot}) for reading by {\tt xmgr, xvgr},
 with data and theoretical curves.
\\ DONE  {\tt search\_file}: write single new search file (default: {\tt done.fresco}) for rereading by {\sc Sfresco} as a combination of {\sc Fresco} {\tt input\_file} and current parameter values.  Data specifications are  however not included.
 \\ LINE {\tt plotfile}: write file (default: {\tt search.plot}) for reading by {\tt xmgr, xvgr},
 with theoretical curves only.
\\ READ {\tt file}: read plot output {\tt file} for further searches, if not:
\\ READ {\tt snapfile}: if the input string {\tt file} contains the string 'snap',
read to the end of previous snap output {\tt snapfile}, for further searches.
\\ ELIM {\tt emin emax}: restrict data scans to this energy range for plots and $\chi^2$ sums
\\ POLES {\tt emin emax}: vary only R-matrix poles in this energy range (others are fixed)
\\ ESCAN {\tt emin emax estep}: scan lab. energy in incident channel, to give
phase shifts in fort.71, fusion \& reaction cross sections in fort.40,
 and S-factors in fort.35 and 75.
This is to look for/at resonances, or at behaviour out of data's energy range,
not for $\chi^2$ fitting.\\
\\ MIN: call {\tt minuit} interactively, to read from stdin.
Return to {\tt SFresco} by {\tt minuit} command {\tt end}.


~\\
%\newpage
A example of this {\tt search\_file} input is:
%\small
\begin{verbatim}
'6hep.in' '6hep.out' 7 1
 &variable kind=1 name='V' kp=1 pline=2 col=1 /
 &variable kind=1 name='r0' kp=1 pline=2 col=2 valmin=.7 valmax=2.0/
 &variable kind=1 name='a' kp=1 pline=2 col=3 /
 &variable kind=1 name='W' kp=1 pline=2 col=4 /
 &variable kind=1 name='rw' kp=1 pline=2 col=5 valmin=.7 valmax=2.0/
 &variable kind=1 name='aw' kp=1 pline=2 col=6 /
 &variable kind=5 name='exptnorm' /
 &data type=0 iscale=2  idir=0 lab=F  abserr=T/
 13.99     565.7     28.3
 17.59     478.19    23.91
 21.22     425.41    21.28
 24.92     338.44    16.93
 28.68     284.71    14.24
 36.48     168.59    8.43
 40.56     137.9     6.86
 44.8      104.75    5.24
 53.       30.0      9.14
 53.94     38.45     2.71
 59.       25.62     2.55
 &
\end{verbatim}
\normalsize
A example of the full {\tt SFresco} input, suitable for batch running, is:
%\small
\begin{verbatim}
search4.in
min
fix 4
migrad
end
q
show
plot
\end{verbatim}
\normalsize
where {\tt search4.in} is the above {\tt search\_file} input file name.

\newpage
\section{Test run: a CCBA calculation}


The following test case calculates the inelastic and transfer cross sections
for the reactions induced by $^{19}$F on $^{28}$Si at 60.0 MeV.   A Triton-Oxygen
cluster structure is assumed for $^{19}$F, with the potentials for the triton
and the oxygen core being folded together to give a diagonal Watanabe
potential for elastic $^{19}$F scattering as well as the couplings to the
inelastic 1/2+ state.
These elastic and inelastic channels are coupled together to all orders,
then a finite-range transfer step is calculated for the $^{16}$O + $^{31}$P channel,
thus performing a CCBA calculation.

{\small
\begin{verbatim}
Test run: CCBA calculation 28Si(19F,16O) 31P, cluster form factors.
NAMELIST
 &FRESCO
 hcm=.10 rmatch=25 rintp=0.50  hnl=0.100  rnl=3.
 jtmin=0. jtmax=80 absend=.01 kqmax=1  thmin=0. thmax=60.0  thinc=2.5
 iter=1   iblock=2 nnu=30 chans=1  listcc=2 smats=1 elab=60.0
 /
 &PARTITION
   namep='19-F' massp=19. zp=9 nex=2 namet='28-SI' masst=28 zt=14 /
 &states Jp=0.5   Bandp=+1 Ep=0.0  cpot=1          Jt=0.0   bandt=+1 Et=0.0 /
 &states Jp=2.5   Bandp=+1 Ep=0.2 /

 &PARTITION
   namep='16-O' massp=16. zp=8 nex=1  namet='31-P' masst=31 zt=15 qval=6.199/
 &states Jp=0.0   Bandp=+1 Ep=0.0  cpot=2          Jt=0.5   bandt=+1 Et=0.0 /
 &partition /

 &pot kp=1 At=28.0     Ap=19.0   Rc=1.2 /
 &pot kp=2 At=31.0     Ap=16.0   Rc=1.35
           V=31.20  rv=1.45    av=0.47    W=15.10   rw=1.270   aw=0.310
           Vso=0.75 rso=1.24    aso=0.37 /
 &pot kp=3 At=19.0     Rc=1.25    ac=0.65
           V=115.00 rv=1.250   av=0.650 Vso=6.30 rso=1.25    aso=0.650 /
 &pot kp=4 At=31.0     Rc=1.25    ac=0.65 V=99.000 rv=1.250   av=0.650
           Vso=6.30 rso=1.25    aso=0.650 /
 &pot kp=5 At=28.0     Rc=1.2
           V=27.00  rv=2.42452 av=0.460   W=11.66   rw=2.14456 aw=0.238 /
 &pot kp=6 At=28.0     Rc=1.25    ac=0.65
           V=160.00 rv=1.07    av=0.72    W=37.0    rw=1.35    aw=0.88
           Vso=10.0 rso=0.85    aso=0.850 /
 &pot  /
 &OVERLAP kn1=1 ic1=1 ic2=2 in=1 kind=0 nn=4 l=0 sn=.5 j=0.5 kbpot=3 be=11.73 isc=1 /
 &OVERLAP kn1=2 ic1=1 ic2=2 in=2 kind=0 nn=4 l=0 sn=.5 j=0.5 kbpot=4 be=17.899 isc=1 /
 &OVERLAP kn1=3 ic1=1 ic2=2 in=1 kind=0 nn=3 l=2 sn=.5 j=2.5 kbpot=3 be=11.53 isc=1 /
 &overlap /
 &COUPLING icto=2   icfrom=1   kind=7 ip1=0 ip2=0 ip3=0/
 &cfp     in=1   ib=1   ia=1   kn=1  a=1.00 /
 &cfp     in=1   ib=2   ia=1   kn=3  a=1.00 /
 &cfp     in=2   ib=1   ia=1   kn=2  a=1.00 /
 &cfp     /
 &COUPLING icto=1   icfrom=2   kind=3 ip1=4 ip2=1 ip3=0 kfrag=6 kcore=5/
\end{verbatim}
}
\newpage

\section{File allocations}
\label{files}
{\small 
\begin{tabular}{|l | c c c | l | l|}
\hline
File & Format & LRECL  & Common & Routines & Use \\
 no. & (Fix/Var) & (bytes) &suffix &  &  \\
\hline
1    &  V &             &     &     SFRESCOX& FRESCOX input when searching\\
2    &  V &             &    &     SFRESCO& Search specification file\\
3    &  V &             &    &     FREADF, FR& Temporary namelist file\\
4    &  F &     80      &    &     INTER & default for external KIND=1,9 couplings\\
     &    &             &    &     POTENT &external potentials\\
5    &  F &     80      & in  &         &   standard input\\
6    &  V &    133      & out  &          &  standard output\\
7    &  F &     80      & smats &     DISPX & elastic S-matrix elements\\
8  \# &  F &   sector    &   (tmp) &     FR,INTER & s/p wfs, channel wfs\\
9  \# &  F &  NLL*NLO*16    &  (tmp)& Q/KERNEL &  complex transfer multipoles\\
10  @ &  V &             &  (tmp)  &     FR,CRISS & S-matrix elements (cfs)\\
11  \# &  F & NLL*NLO*8 &  (tmp) & Q/KERNEL &  real transfer multipole\\
12  \# &  V &NLL*NLO*16 & (tmp)  &   KERNEL/SOURCE&transfer kernels\\
13    &  F &    80      & xst  &      FR   &    total cross sections for each Elab\\
14  \#  & V &             &    &   INTER/CPAIR &interaction potentials\\
15    &  V &           & (tmp)  &       FR   &   local equivalent potent\\
16    &  F &     80      & xsec  &       CRISS &  tables of cross sections\\
17    &  F &     80      & swf  &       FR  &    output scattering waves\\
18  \# &  V &             & (tmp)  &       FR  &    wfns of 'best' iterate\\
19    &  F &    N      &    &     FR  &   Local couplings\\
     &  & & & &\\
20--33&    &            &    &     Available for users & (eg bound states, amplitudes, potentials)\\
     &  & & & &\\
34    &  F &    80      &    &     POTENT &   output potentials\\
35    &  F &    80      & sf  &     FR & Astrophysical S-factors for Ecm\\
36    &    &            &    &     CRISS & output scattering AMPL amplitudes\\
37    &    &            &    &     CRISS & output scattering FAM amplitudes\\
38    &  F &    80      &    &     DISPX &    cross sections for each J/pi\\
39    &  F &    80      & efus  &      FR   &    2 cross sections for each Elab\\
40    &  F &    80      & fus  &      FR   &    all cross sections for each Elab\\
41    &  F &    80      &    &     SOURCE &   source terms at each iteration\\
42    &  F &    80      &    &     SOURCE &   bin wavefunctions for each E\\
43    &  F &    80      &    &     INFORM &   bin phase shifts as k functions\\
44    &  F &    80      & phe  &     INFORM &   bin phase shifts as E functions\\
45    &  F &    80      &    &     ERWIN   &  scat phase shift as E functions\\
46    &  F &    80      &    &     INFORM &   bs wave functions \& Whit ratios\\
47    &  F &    80      &   mek &           &   reduced matrix elements \\
48  \#  & V &   133      &    &       FR  & concurrency log file\\
49  \#  & V &        &    &     FR,CRISS & S-matrix elements concurrent$\rightarrow$10\\
\hline
\end{tabular}

\begin{tabular}{|l | c c c | l | l|}
\hline
File & Format & LRECL  & Access & Routines & Use \\
 no. & (Fix/Var) & (bytes) & (Seq/Ran) &  &  \\
\hline
50  \#  & V &   133  &    &    &    tmp standard output concurrent$\rightarrow$6\\
51,52 \#  & V &   133  &    &  & unwanted standard output for node N$>$0\\
53    &   V &            &  mel  &     WRITEMEL & mel file for sturmx(x)\\
54    &   V &            &   spec  &     WRITESPEC & spec file for sturmx(x)\\
55    &   F &            &    &     INFORM & Single-particle wave functions\\
56    &   F &            &  fusl  &     FR & Jtotal fusion \& reaction cross\\
57    &   F &            &   cdcc &     FR & Output of CDCC amplitudes\\
58    &   F &            &  wfs  &     INFORM & Bound state wave functions \\
59    &   F &            &   vrt &     INFORM & Bound state vertex functions \\
60-62 &   F &            &  imp  &     RMATRIX & Trace of R-matrix calculations\\
63    &   F &            &    &     WRITEMEL& Asymptotic coupling coefficients\\
64    &   F &            &    &     RMATRIX& Coupled eigenstates\\
65 \# &  F  &   NLO*2*8  & (tmp)   &     QNN  &  two-particle multipoles\\
66    &  V  &NLL*NLO*16  & &     INTER & KIND=1 nonlocal formfactor\\
70    &   F &            &    &     RMATRIX& Basis state wfs\\
71    &   F &    132     &phs &     FR   &  20 pw phase shifts as Elab functions\\
79+CP  &  V &\{N,NLN*NLO\}*16 & (tmp)  &   INTER/KERNEL &(non)local partial-wave couplings\\
87    &   V &            &   pnl &     KERNEL & Non-local couplings\\
88    &   F &            &    &     RMATRIX& Bound state wfs\\
89    &   F &            &    &     MULTIP& Folded coupling potentials\\
90    &  V &           &veff &       FR   &   average local equivalent potential\\
91    &  V &           &vpol &       FR   &   average LEP for rereadingl\\
92    &   F &            &    &     FR& Partial fusion cross sections\\
% 93    &   F &            &    &     FR& Data for PVM distributions\\
99     & F &    80   & &     CRISS   &  input J/L rescaling factors\\
105     & V &         & &     FCN   &  $\chi^2$ progress during fitting\\
200 & F & 80     &xsel &     CRISS   & Elastic cross section if not fort.201 \\
201-210 & F & 80     &xs[1-9] &     CRISS   & Cross sections (cf 14) for states 1-10\\
301 & V &      & &     CDCIN   & New Fresco input \\
302 & V &      & &     CDCIN   & New Fresco input  (temp) \\
303 & V &      & &     SFRESCO   & Input search file \\
304 & V &      & &     SFRESCO   & Output plot file \\
305 & V &      & &     CDCIN   & New Fresco input from cdcin, col format \\
306 & V &      & &     SFRESCO/FRXX0   & Input Fresco file \\
307 & V &      & &     SFRESCO/FRXX0   & Initial Output Fresco file \\
308 & V &      & &     SFRESCO/FRXX0   & Main Output Fresco file \\
470+$i$ & V &       & &     erwinmpi   & Log files from MPI process $i$ \\
\hline
\end{tabular}
}
% no. & (Fix/Var) & (bytes) & (Seq/Ran) &  &  \\

where  for MACH=8 we have\\
\#  = distributed file: distinct file name on each node\\
@  = shared concurrent file: single file, with writes from each node

The use of files 19 and 35 have recently changed.

\newpage
\section{KIND=1 Spin Transfer Couplings}
\label{spintransfers}
The definition of the KIND=1 spin transfer couplings is not given
in the Computer Physics Reports article, so in versions after March 1998 these
are redefined for IP3=0 or 1, and new Racah algebra factors included.

\bigskip

We want to calculate the coupling interactions of the monopole operator
$   {\bf S}([\ell,s_p]s_t,s_t)  $, where $s_p$ is the spin transfer of the
projectile $I_p$,  $s_t$ is the spin transfer of the
target $I_t$, and $\ell$ is the orbital angular momentum transfer.
 %
These coupled operators are defined following in Bohr \& Mottelson, Vol.
1, section 1A-5c, as
\begin{equation}
(F_{\lambda_1} G_{\lambda_2})_{\lambda \mu}
  = \sum_{\mu_1 \mu_2} \langle \lambda_1 \mu_1 ,  \lambda_2 \mu_2
  |  \lambda \mu \rangle F_{\lambda_1\mu_1} G_{\lambda_2\mu_2} \ ,
\end{equation}
applied for the case of $F_{s_t} = [\ell,s_p]s_t$ and $ G_{s_t}=s_t$.
The overall ${\bf S}$ operator is a monopole (scalar), for which the tensor
product is thus of the kind
\begin{equation}
(F_{\lambda} G_{\lambda})_{00}
  = (2\lambda+1)^{-1/2} \sum_{\mu} (-1)^{\lambda-\mu} ~ F_{\lambda\mu}
  G_{\lambda-\mu} \ .
\end{equation}
This differs from common definitions (eg of Satchler) by a factor of
$ (-1)^\lambda (2\lambda+1)^{-1/2} $.

\bigskip

Reduced matrix elements are defined everywhere in {\sc Fresco} by:
\begin{equation}
% <j_f m_f | \hat{O}_{j m} | J_i m_i > = { <j_i m_i \ j m |\ j_f m_f > \over
% \hat{j}_f } < j_f || \hat{O}_j || j_i>
\langle j_f m_f | \hat{O}_{\lambda \mu} | J_i m_i \rangle
 = (2j_f+1)^{-1/2} ~ \langle j_i m_i ,\lambda \mu | j_f m_f \rangle ~
\langle j_f || \hat{O}_\lambda || j_i\rangle
\end{equation}
\bigskip

The  matrix elements of this operator are
\begin{eqnarray}
&& \langle(LI_p)J,I_t;J_TM_T | ~{\bf S}([\ell,s_p]s_t,s_t) ~| (L'I_p')J',I_t';J_TM_T \rangle
   \nonumber \\
& = &  (-1)^{s_t+J_T+J'+I_t}
   \left\{ \begin{array}{ccc}
    J'  & I_t' & J_T \\
    I_t & J    & s_t \\
    \end{array} \right\}   \hat{J} \hat{J'}
    \left\{ \begin{array}{ccc}
    L' & I_p' & J'  \\
    \ell & s_p & s_t  \\
    L & I_p & J'  \\
    \end{array} \right\}  ~
     \frac{1}{\sqrt{4\pi}} ~ \hat{\ell} \hat{L} \langle L 0 \ \ell 0 | L'0 \rangle   \nonumber \\
    &  &   \times \langle I_p||s_p||I_p'\rangle \langle I_t||s_t||I_t'\rangle
\end{eqnarray}
In using KIND=1 couplings with IP3=0, the first line of these factors is generated automatically. \\
The product of the
reduced matrix elements for the intrinsic nuclear states,
$\langle I_p||s_p||I_p'\rangle \langle I_t||s_t||I_t'\rangle$,
 has to be included explicitly in the factor FSCALE,
 or in the radial form factors.
The radial shapes have to be read in from data files.\\

With IP3=2, implying jlmP input for target-only couplings,
\begin{itemize}
\item
 the projectile diagonal reduced matrix element
$\langle I_p||s_p=0||I_p\rangle = \hat{I_p}$ is supplied by Fresco,
\item
the factor $1/\sqrt{4\pi}$ is omitted above,
\item
and an additional symmetric factor of
%$\sqrt{\hat{I_t'}\hat{I_t}}$
$((2I_t'+1)(2I_t+1))^{1/4}$
is supplied to allow the
monopole radial form factors to have their physical
values for any target spin.
\item
The forward and reverse form factors should be identical for Hermitian couplings.

\end{itemize}
With IP3=3, implying a similar input for projectile-only couplings,
\begin{itemize}
\item
 the target diagonal reduced matrix element
 $\langle I_t||s_t=0||I_t\rangle = \hat{I_t}$ is supplied by Fresco,
 \item
 the factor $1/\sqrt{4\pi}$ is omitted above,
 \item
 and an additional symmetric factor of $((2I_t'+1)(2I_t+1))^{1/4}$
 is supplied to allow the monopole radial form factors to have their physical
 values for any target spin.
 The forward and reverse form factors should be identical for Hermitian couplings.
\end{itemize}

%~\\
%The reduced matrix elements are defined by:
%\begin{equation}
%<j_f m_f | \hat{O}_{j m} | J_i m_i > = { <j_i m_i \ j m |\ j_f m_f > \over
%\hat{j}_f } < j_f || \hat{O}_j || j_i>
%\end{equation}
%
\end{document}
